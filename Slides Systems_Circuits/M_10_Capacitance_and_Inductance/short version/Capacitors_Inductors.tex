\documentclass[10pt,makeidx]{beamer}
\usetheme{Boadilla}
%\usecolortheme{seahorse}
% \usefonttheme{structurebold}
\usefonttheme{serif}

\usepackage{etex}
% \usepackage{helvet}
\usepackage{amsmath, amssymb}
\usepackage{color}
\usepackage{asymptote}
\usepackage{mathrsfs}
\usepackage{dsfont}
\usepackage{makeidx}
\usepackage{multido}
\usepackage{adjustbox}
\usepackage{cancel}

\usepackage{pst-sigsys,pst-plot,pstricks-add}
\usepackage{pst-pdf}

\usepackage[american voltages, american currents,siunitx]{circuitikz}
%\sisetup{load=derived} % loading \siemens


\makeatletter
% create the shape
\pgfcircdeclarebipole{}{\ctikzvalof{bipoles/interr/height 2}}{spst}{\ctikzvalof{bipoles/interr/height}}{\ctikzvalof{bipoles/interr/width}}{

    \pgfsetlinewidth{\pgfkeysvalueof{/tikz/circuitikz/bipoles/thickness}\pgfstartlinewidth}

    \pgfpathmoveto{\pgfpoint{\pgf@circ@res@left}{0pt}}
    \pgfpathlineto{\pgfpoint{.6\pgf@circ@res@right}{\pgf@circ@res@up}}
    \pgfusepath{draw}   
}

% make the shape accessible with nice syntax
\def\pgf@circ@spst@path#1{\pgf@circ@bipole@path{spst}{#1}}
\tikzset{switch/.style = {\circuitikzbasekey, /tikz/to path=\pgf@circ@spst@path, l=#1}}
\tikzset{spst/.style = {switch = #1}}
\makeatother



%
%%%=== showexpl ======================================================
%\usepackage{showexpl}
%\lstdefinelanguage{PSTSigSys}{%
%   morekeywords={psaxeslabels,pstick,pssignal,psstem,pszero,pspole,pscircleop,psframeop,%
%                 psring,psdisk,psdiskc,psldots,ldotsnode,psblock,psfblock,psadaptive,psknob,%
%                 psusampler,psdsampler,nclist,ncstar,psBraceUp,psBraceDown,psBraceLeft,psBraceRight},
%   morekeywords={dotnode},
%   sensitive=false
%}
%\lstset{%
%   explpreset={numbers=left,numberstyle=\tiny,numbersep=5pt},%
%   basicstyle=\ttfamily\small,%
%   rframe=,%
%   frame=single,%
%   frameround=tttt,%
%   aboveskip=\baselineskip,%
%   belowskip=\baselineskip,%
%   backgroundcolor=\color{Tangerine!10},%
%   language=PSTricks,%
%   alsolanguage=PSTSigSys,%
%   keywordstyle=\color{ForestGreen},%
%   commentstyle=\color{DarkBrown}%
%}
%%%=== end showexpl ==================================================
%
%%%=== colors ========================================================
%\definecolor{Salmon}{RGB}{178,51,51}
%\definecolor{BrickRed}{RGB}{233,49,16}
%\definecolor{TealBlue}{RGB}{16,83,165}
%\definecolor{ForestGreen}{RGB}{10,112,43}
%\definecolor{Tangerine}{RGB}{244,176,108}
%\definecolor{DarkBrown}{RGB}{130,65,0}
%%%=== end colors ========
%


\definecolor{links}{HTML}{2A1B81}
\hypersetup{colorlinks,linkcolor=,urlcolor=links}

\def\nn{\nonumber}

\definecolor{links}{HTML}{2A1B81}
\hypersetup{colorlinks,linkcolor=,urlcolor=links}


\title[]{Capacitance and Inductance}
\author[\textcolor{blue}{}]{\textcolor{violet}{System and Circuits. Topic-4: Filters}\\\vspace*{3mm}\textsf{{\small \textcolor{blue}{Pablo M. Olmos, olmos@tsc.uc3m.es\\ Emilio Parrado, emipar@tsc.uc3m.es\\ Rocio Arroyo Valles, marrval@tsc.uc3m.es}}}
}
%\date[]{{\small }}
\institute{\textcolor{white}{\{olmos,emipar\}@tsc.uc3m.es}}
%\logo{\includegraphics[scale=0.2]{Figures/uc3m-logo.pdf}}



\AtBeginSection[]
{
  \begin{frame}<beamer>{Index}
    \tableofcontents[currentsection,currentsubsection]
  \end{frame}
}

\begin{document}
\frame{
\titlepage
\thispagestyle{empty}
\begin{center}
\includegraphics[scale=0.15]{Figures/uc3m-logo.pdf}
\end{center}
}

\section{Introduction}

\frame{
\begin{itemize}
\item Capacitors and Inductors are passive two-terminal electrical components used to store energy in an electric/magnetic field. %They are able to store electrical charge for a long time after power has been turned off in a circuit. 
\item  Both are basic components used in electronics where \textbf{current and voltage change} with time, due to their ability  to delay and reshape alternating currents/voltages.
\item \textbf{Analog filters}: circuits containing resistors, capacitors, inductors y active sources that change with time.
\end{itemize}

\begin{center}
\begin{circuitikz} \draw[black]
(0,0) to[C] (2,0) 
(3,0) to[L] (5,0)
(1,-1) node[]{Capacitor}
(4,-1) node[]{Inductor};
\end{circuitikz}
\end{center}

%\begin{block}{}
%In the following, we move from time-constant voltage and current sources  to time-varying sources. 
%\end{block}


}

\section{The capacitor}

\frame{
\frametitle{The capacitor}

\begin{itemize}
\item Composed by two electrical conductors separated by a dielectric material.
\item The application of of a voltage to the terminals creates a displace of charge within the dielectric. 
\item The displacement current is indistinguishable from a conduction current. We simply refer to it as the current across the capacitor.
\item The current across a capacitor is proportional to the rate at which  the voltage varies with time:
\begin{align}\nn
i(t)&=C\frac{d v(t)}{d t} ~~~~\text{ (Fundamental equation of the capacitor) }
%\Rightarrow v(t)&=\int_{-\infty}^{t}\frac{1}{C}i(\tau)\text{d}\tau
%=v(t_0)+\int_{t_0}^{t}\frac{1}{C}i(\tau)\text{d}\tau
\end{align}
\end{itemize}

\begin{columns}
\begin{column}{0.3\textwidth}
\begin{center}
\begin{circuitikz} \draw[black]
(0,0) to[C=$C$,i>=$i$,v=$v$] (2,0);
\end{circuitikz}
\end{center}\end{column}
\begin{column}{0.5\textwidth}
\begin{block}{}
$C$ is called the capacity and it is measured in Farads (F).
\end{block}
\end{column}
\end{columns}



}

\frame{




\begin{columns}
\begin{column}{0.3\textwidth}
\begin{center}
\begin{circuitikz} \draw[black]
(0,0) to[C=$C$,i>=$i$,v=$v$] (2,0);
\end{circuitikz}
\end{center}\end{column}
\begin{column}{0.5\textwidth}
\begin{align}\nn
i(t)&=C\frac{d v(t)}{d t} 
%\Rightarrow v(t)&=\int_{-\infty}^{t}\frac{1}{C}i(\tau)\text{d}\tau
%=v(t_0)+\int_{t_0}^{t}\frac{1}{C}i(\tau)\text{d}\tau
\end{align}
\end{column}
\end{columns}

\begin{alertblock}{Remarks:}
\begin{enumerate}
\item If the voltage across the terminals is constant, the current is zero. Thus, \textbf{a capacitor behaves as an open circuit in the presence of a constant voltage}.
\item The voltage cannot change instantaneously across the capacitor. Such a change would produce infinite current, physically impossible. For any time $t_0$:
\begin{align*}
v(t_0^{-})=v(t_0)=v(t_0^{+})
\end{align*}
\end{enumerate}
\end{alertblock}

}

\frame{
\frametitle{The capacitor}
\begin{columns}
\begin{column}{0.3\textwidth}
\begin{center}
\begin{circuitikz} \draw[black]
(0,0) to[C=$C$,i>=$i(t)$,v=$v(t)$] (2,0);
\end{circuitikz}
\end{center}\end{column}
\begin{column}{0.5\textwidth}
\begin{align}\nn
i(t)&=C\frac{d v(t)}{d t}
\end{align}
\end{column}
\end{columns}

\vspace{0.5cm}
\begin{block}{}
If we know the current across the capacitor along the time $i(t)$, we can compute the voltage $v(t)$ as follows:
\begin{align}\nn
v(t)&=\frac{1}{C}\int_{-\infty}^{t}i(\tau)\text{d}\tau=\frac{1}{C}\int_{t_0}^{t}i(\tau)\text{d}\tau+v(t_0)
\end{align}
\end{block}

%\begin{exampleblock}{\textbf{Power}}
%$p(t)=v(t)i(t)=Cv(t)\frac{d v(t)}{d t}\Rightarrow$ $p(t)$ can be positive (we are storing energy) or negative (the energy stored is been given back to the circuit).
%\end{exampleblock}


}

%\frame{
%\frametitle{Example}
%The voltage pulse described by the following equations is applied to a $0.5-\mu F$ capacitor: 
%
%\begin{align*}
%v(t)=\left\{
%\begin{array}{cc}
%0 & t\leq 0\\
%4t~V & 0<t\leq 1\\
%4 e^{-(t-1)}~V & 1<t \leq \infty
%\end{array}
%\right.
%\end{align*}
%
%\begin{itemize}
%\item Derive the expression for the capacitor current.
%\item Sketch the voltage and current as functions of time.
%\end{itemize}
%
%}

\frame{
\frametitle{Power and Energy in the Capacitor}
\begin{block}{}
\begin{columns}
\begin{column}{0.2\textwidth}
\begin{center}
\begin{circuitikz} \draw[black]
(0,0) to[C=$C$,i>=$i(t)$,v=$v(t)$] (2,0);
\end{circuitikz}
\end{center}\end{column}
\begin{column}{0.2\textwidth}
\begin{align}\nn
i(t)&=C\frac{d v(t)}{d t}
\end{align}
\end{column}
\begin{column}{0.5\textwidth}
\begin{align}\nn
v(t)&=\frac{1}{C}\int_{t_0}^{t}i(\tau)\text{d}\tau+v(t_0)
\end{align}
\end{column}
\end{columns}
\end{block}

From the definition of power
\begin{align*}
p(t)=v(t)i(t)=Cv(t)\frac{dv(t)}{dt}
\end{align*}
or
\begin{align*}
p(t)=v(t)i(t)=i(t)\left[\frac{1}{C}\int_{t_0}^{t}i(\tau)\text{d}\tau+v(t_0)\right]
\end{align*}

\begin{alertblock}{}
If $p(t)>0$  the capacitor is storing energy. If $p(t)<0$, it means the capacitor is delivering energy to the circuit.
\end{alertblock}

}

\frame{
\frametitle{Power and Energy in the Capacitor}
The energy in the capacitor at any time can be computed as follows:
\begin{align*}
p(t)=\frac{d w(t)}{d(t)}\Rightarrow w(t)=\int_{-\infty}^{t} p(t) dt=\int_{-\infty}^{t}  v(t) C\frac{d v(t)}{\cancel{d t}} \cancel{dt}=C\frac{v^2(t)}{2}
\end{align*}

\begin{exampleblock}{}
The energy $w(t)$ in the capacitor is always positive!
\end{exampleblock}
}
%
%\frame{
%\frametitle{Example (cont.)}
%
%The voltage pulse described by the following equations is applied to a $0.5-\mu F$ capacitor: 
%
%\begin{align*}
%v(t)=\left\{
%\begin{array}{cc}
%0 & t\leq 0\\
%4t~V & 0<t\leq 1\\
%4 e^{-(t-1)}~V & 1<t \leq \infty
%\end{array}
%\right.
%\end{align*}
%
%\begin{itemize}
%\item Derive the expression for the power $p(t)$ and the energy $w(t)$ in the capacitor.
%\item Specify the interval of time when energy is being stored in the capacitor.
%\item Specify the interval of time when energy is being delivered by the capacitor.
%\item Evaluate the integrals 
%\begin{align*}
%\int_{0}^{1} p(t) dt \qquad \int_{1}^{\infty} p(t) dt
%\end{align*}
%and comment their significance.
%\end{itemize}
%
%}

\frame{
\frametitle{Capacitors in parallel/series}

\begin{circuitikz} \draw[black]
(-2,0) to[short,o-] (5,0)
(-2,2) to[short,o-] (5,2)
(0,0) to[C=$C_1$] (0,2)
(2,0) to[C=$C_2$] (2,2)
(5,0) to[C=$C_N$] (5,2)
(3,1) node[]{\ldots}
(7,0)to[C,o-o](7,2)
(9,1) node[]{$C_{eq}=\sum_{i}C_i$};
\end{circuitikz}
\vspace{1.5cm}

\begin{circuitikz} \draw[black]
(0,0) to[C=$C_1$,o-] (1.5,0)
to[C=$C_2$] (3,0)
(4,0) node[]{$\ldots$}
(5,0) to[C=$C_N$,-o] (6.5,0)
(8,0)to[C,o-o](10,0)
(9,1) node[]{$C_{eq}=\left(\sum_{i}\frac{1}{C_i}\right)^{-1}$};
\end{circuitikz}



}


\section{The Inductor}

\frame{
\frametitle{The Inductor}

\begin{itemize}
\item Inductors are circuit elements based on phenomena associated with magnetic fields.
\item The source of the magnetic field is charge in motion (current).
\item If the current is varying with time, the magnetic field is varying with time, which induces a voltage in any conductor linked by the field.
\item The voltage drop across the terminals of the inductor capacitor is proportional to the rate at which  the current varies with time:
\begin{align}\nn
v(t)&=L\frac{d i(t)}{d t} \text{ (Fundamental equation of the inductor) }
%\Rightarrow v(t)&=\int_{-\infty}^{t}\frac{1}{C}i(\tau)\text{d}\tau
%=v(t_0)+\int_{t_0}^{t}\frac{1}{C}i(\tau)\text{d}\tau
\end{align}
\end{itemize}

\begin{columns}
\begin{column}{0.3\textwidth}
\begin{center}
\begin{circuitikz} \draw[black]
(0,0) to[L=$L$,i>=$i$,v=$v$] (2,0);
\end{circuitikz}
\end{center}\end{column}
\begin{column}{0.5\textwidth}
\begin{block}{}
$L$ is called the inductance and it is measured in Henrys (H).
\end{block}
\end{column}
\end{columns}



}

\frame{




\begin{columns}
\begin{column}{0.3\textwidth}
\begin{center}
\begin{circuitikz} \draw[black]
(0,0) to[L=$L$,i>=$i$,v=$v$] (2,0);
\end{circuitikz}
\end{center}\end{column}
\begin{column}{0.5\textwidth}
\begin{align}\nn
v(t)&=L\frac{d i(t)}{d t}
\end{align}
\end{column}
\end{columns}

\begin{alertblock}{Remarks:}
\begin{enumerate}
\item If the current across the terminals is constant, the voltage is zero. Thus, \textbf{a capacitor behaves as a short circuit in the presence of a constant current}.
\item The current cannot change instantaneously across the inductor. Such a change would produce infinite voltage drop across the terminals, physically impossible. For any time $t_0$:
\begin{align}\nn
i(t_0^{-})=i(t_0)=i(t_0^{+})
\end{align}
\end{enumerate}
\end{alertblock}

}

\frame{
\frametitle{The inductor}
\begin{columns}
\begin{column}{0.3\textwidth}
\begin{center}
\begin{circuitikz} \draw[black]
(0,0) to[L=$L$,i>=$i$,v=$v$] (2,0);
\end{circuitikz}
\end{center}\end{column}
\begin{column}{0.5\textwidth}
\begin{align}\nn
v(t)&=L\frac{d i(t)}{d t}
\end{align}
\end{column}
\end{columns}


\vspace{0.5cm}
\begin{block}{}
If we know the voltage in the inductor $v(t)$ along  time, we can compute the current $i(t)$ as follows:
\begin{align}\nn
i(t)&=\frac{1}{L}\int_{-\infty}^{t}v(\tau)\text{d}\tau=\frac{1}{L}\int_{t_0}^{t}v(\tau)\text{d}\tau+i(t_0)
\end{align}
\end{block}

%\begin{exampleblock}{\textbf{Power}}
%$p(t)=v(t)i(t)=Cv(t)\frac{d v(t)}{d t}\Rightarrow$ $p(t)$ can be positive (we are storing energy) or negative (the energy stored is been given back to the circuit).
%\end{exampleblock}


}


\frame{
\frametitle{Power and Energy in the Inductor}
\begin{block}{}
\begin{columns}
\begin{column}{0.2\textwidth}
\begin{center}
\begin{circuitikz} \draw[black]
(0,0) to[L=$L$,i>=$i$,v=$v$] (2,0);
\end{circuitikz}
\end{center}\end{column}
\begin{column}{0.2\textwidth}
\begin{align}\nn
v(t)&=L\frac{d i(t)}{d t}
\end{align}
\end{column}
\begin{column}{0.5\textwidth}
\begin{align}\nn
i(t)&=\frac{1}{L}\int_{t_0}^{t}v(\tau)\text{d}\tau+i(t_0)
\end{align}
\end{column}
\end{columns}
\end{block}

From the definition of power
\begin{align*}
p(t)=v(t)i(t)=Li(t)\frac{di(t)}{dt}
\end{align*}
or
\begin{align*}
p(t)=v(t)i(t)=v(t)\left[\frac{1}{L}\int_{t_0}^{t}v(\tau)\text{d}\tau+i(t_0)\right]
\end{align*}

\begin{alertblock}{}
If $p(t)>0$  the inductor is storing energy. If $p(t)<0$, it means the inductor is delivering energy to the circuit.
\end{alertblock}

}

\frame{
\frametitle{Power and Energy in the Inductor}
The energy in the inductor at any time can be computed as follows:
\begin{align*}
p(t)=\frac{d w(t)}{d(t)}\Rightarrow w(t)=\int_{-\infty}^{t} p(t) dt=\int_{-\infty}^{t}  i(t) L\frac{i v(t)}{\cancel{d t}} \cancel{dt}=L\frac{i^2(t)}{2}
\end{align*}

\begin{exampleblock}{}
The energy $w(t)$ in the inductor is always positive!
\end{exampleblock}
}


\frame{
\frametitle{Inductors in parallel/series}

\begin{circuitikz} \draw[black]
(-2,0) to[short,o-] (5,0)
(-2,2) to[short,o-] (5,2)
(0,0) to[L=$L_1$] (0,2)
(2,0) to[L=$L_2$] (2,2)
(5,0) to[L=$L_N$] (5,2)
(3,1) node[]{\ldots}
(7,0)to[L,o-o](7,2)
(9,1) node[]{$L_{eq}=\left(\sum_{i}\frac{1}{L_i}\right)^{-1}$};
\end{circuitikz}
\vspace{1.5cm}

\begin{circuitikz} \draw[black]
(0,0) to[L=$L_1$,o-] (1.5,0)
to[L=$L_2$] (3,0)
(4,0) node[]{$\ldots$}
(5,0) to[L=$L_N$,-o] (6.5,0)
(8,0)to[L,o-o](10,0)
(9,1) node[]{$L_{eq}=\sum_{i}L_i$};
\end{circuitikz}



}
%
%\frame{
%\frametitle{Example}
%The independent current source in the circuit shown generates zero current for $t<0$ and a pulse $i(t)=10 t \text{e}^{-5t}$.
%
%\begin{center}
%\vspace{0.5cm}
%\begin{circuitikz} \draw[black]
%(0,0) to[I=$i(t)$]  (0,2)
%to[short](2,2)
%to[L=$100mH$,v=$v(t)$] (2,0)
%to[short] (0,0);
%\end{circuitikz}
%\end{center}
%
%
%\begin{itemize}
%\item Sketch the current waveform (computing the derivative w.r.t. helps).
%\item At what instant of time is the current maximum?
%\item Express the voltage $v(t)$ across the terminals of the 100-mH inductor as a function of time.
%\item Sketch the voltage waveform. At what instant does $v(t)$ change polarity?
%\end{itemize}
%
%}
%
%\frame{
%\frametitle{Example (cont.)}
%The independent current source in the circuit shown generates zero current for $t<0$ and a pulse $i(t)=10 t \text{e}^{-5t}$.
%
%\begin{center}
%\vspace{0.5cm}
%\begin{circuitikz} \draw[black]
%(0,0) to[I=$i(t)$]  (0,2)
%to[short](2,2)
%to[L=$100mH$,v=$v(t)$] (2,0)
%to[short] (0,0);
%\end{circuitikz}
%\end{center}
%
%
%\begin{itemize}
%\item Plot the power $p(t)$ and energy $w(t)$ in the inductor along time.
%\item In what time interval is energy being stored in the inductor?
%\item In what time interval is energy being extracted from the inductor?
%\item What is the maximum energy stored in the inductor?
%\item Evaluate the integrals
%\begin{align*}
%\int_{0}^{0.2} p(t) dt \qquad \int_{0.2}^{\infty} p(t) dt
%\end{align*}
%and comment on their significance.
%\end{itemize}
%
%}
%



%
%\section{RL circuits}
%
%
%\frame{
%\begin{itemize}
%\item We are now in a position to determine the currents and voltages that arise when energy is either released or acquired by an Inductor in response to abrupt change in a constant voltage or current source.
%\item We now focus on the RL circuit: a single inductor, a resistor network and a source.
%\end{itemize}
%
%
%
%\begin{center}
%\begin{circuitikz}[] \draw[black]
%(-2,0) to[I=$I_s$] (-2,2)
%to[short](-0.5,2)
%(-0.5,0) to[R=$R_s$] (-0.5,2)
%to[switch] (2,2)
%to[L=$L$,i>=$i_L$] (2,0)
%(2,2) to[switch] (4,2)
%to[R=$R$] (4,0)
%to[short] (-2,0);
%%(6,0) to[V=$V_s$] (6,2)
%%to[switch] (8,2)
%%to[C=$C$] (8,0)
%%(8,2) to[short] (10,2)
%%to[R=$R$] (10,0)
%%to[short] (6,0)
%%%(2,3) node[] {$t<t_0$}
%%%(8,3) node[] {$t\geq t_0$}
%%(6,0) node[ground] {}
%%(8,2.25) node[] {A};
%\end{circuitikz}
%\end{center}
%
%}
%
%
%\frame{
%We divide our analysis in two steps:
%\begin{enumerate}
%\item We consider the currents and voltages that arise when stored energy in an inductor is suddenly released to a resistive network. This is called the \textbf{natural response} of the circuit.
%\item Then we consider the currents and voltages that aires when energy is being acquired by the inductor due to a sudden application of a constant voltage or current source. This is called the \textbf{natural response} of the circuit.
%\end{enumerate}
%
%
%\begin{columns}
%\begin{column}{0.3\textwidth}
%\begin{center}
%\begin{center}
%\begin{circuitikz} \draw[black]
%(2,2) to[L=$L$,i>=$i_L$] (2,0)
%(2,2) to[cspst=$t_0$] (4,2)
%to[R=$R$] (4,0)
%to[short] (2,0)
%(3,3) node[] {$i_L(t_0^{-})=I_0$}
%(3,-1) node[]{\textbf{Natural response.}};
%\end{circuitikz}
%\end{center}
%\end{center}\end{column}
%\begin{column}{0.7\textwidth}
%\begin{center}
%\begin{circuitikz}[] \draw[black]
%(-2,0) to[I=$I_s$] (-2,2)
%to[short](-0.5,2)
%(-0.5,0) to[R=$R_s$] (-0.5,2)
%to[cspst=$t_0$] (2,2)
%to[L=$L$,i>=$i_L$] (2,0)
%(2,2) to[short] (4,2)
%to[R=$R$] (4,0)
%to[short] (-2,0)
%(1,3) node[] {$i_L(t_0^{-})=0$}
%(1,-1) node[]{\textbf{Step response.}};
%%(6,0) to[V=$V_s$] (6,2)
%%to[switch] (8,2)
%%to[C=$C$] (8,0)
%%(8,2) to[short] (10,2)
%%to[R=$R$] (10,0)
%%to[short] (6,0)
%%%(2,3) node[] {$t<t_0$}
%%%(8,3) node[] {$t\geq t_0$}
%%(6,0) node[ground] {}
%%(8,2.25) node[] {A};
%\end{circuitikz}
%\end{center}
%
%\end{column}
%\end{columns}
%
%
%
%}
%
%
%\frame{
%\frametitle{Natural response of the RL circuit}
%
%For simplicity, we close the switch at $t_0=0$ s. 
%\begin{center}
%\begin{circuitikz} \draw[black]
%(2,2) to[L=$L$,i>=$i_L$,v<=$V_L$] (2,0)
%(2,2) to[cspst=$t_0$] (4,2)
%to[R=$R$] (4,0)
%to[short] (2,0)
%(3,3) node[] {$i_L(t_0^{-})=I_0$}
%(3,-1) node[]{\textbf{Natural response.}};
%\end{circuitikz}
%\end{center}
%
%\textbf{Kirchhoff's voltage law ($t\geq0$):}
%\begin{align}\nn
%V_L(t)+i_LR=0\Rightarrow i_L R+L\frac{di_L(t)}{dt}=0
%\end{align}
%}
%
%\frame{
%\frametitle{Natural response of the RL circuit}
%
%The differential equation can be simply compute using the variable separation method:
%\begin{align}\nn
%i_L R+L\frac{di_L(t)}{dt}=0 
%\end{align}
%%
%%Integrating at both sides
%%\begin{align}\nn
%%&\int_{V_C(0)}^{V_C(t)} -\frac{dx}{x}=\frac{1}{RC}\int_{0}^{t} dy\Rightarrow \log(\frac{V_C(0)}{V_C(t)})=\frac{1}{RC}t
%%\end{align}
%%Therefore
%\begin{exampleblock}{Current through the Inductor}
%\begin{align}\nn
%i_L(t)=i_L(0)\text{e}^{-\frac{R}{L}t}=I_0\text{e}^{-\frac{R}{L}t}\quad t\geq 0
%\end{align}
%\end{exampleblock}
%
%\begin{block}{Voltage in the Inductor}
%\begin{align}\nn
%V_L(t)=L\frac{d i(t)}{d t}=-I_0R\text{e}^{-\frac{R}{L}t}
%\end{align}
%\end{block}
%
%}
%
%
%\frame{
%\frametitle{Step response of the RC circuit}
%
%\begin{center}
%\begin{circuitikz}[] \draw[black]
%(-2,0) to[I=$I_s$] (-2,2)
%to[short](-0.5,2)
%(-0.5,0) to[R=$R_s$,i<=$i_{R_s}$] (-0.5,2)
%to[cspst=$t_0$] (2,2)
%to[L=$L$,i>=$i_L$,v<=$V_L$] (2,0)
%(2,2) to[short] (4,2)
%to[R=$R$,i>=$i_R$] (4,0)
%to[short] (-2,0)
%(1,3) node[] {$i_L(t_0^{-})=0$}
%(1,-1) node[]{\textbf{Step response.}};
%\end{circuitikz}
%\end{center}
%
%\begin{align}\nn
%&I_s=i+i_{R_s}+i_L+i_R\Rightarrow I_s=\frac{V_L}{R_s}+i_L+\frac{V_L}{R}\\\nn
%&I_s=\frac{1}{R_s}L\frac{di_L(t)}{dt}+i_L+\frac{1}{R}L\frac{di_L(t)}{dt}=L(R^{-1}_s+R^{-1})\frac{di_L(t)}{dt}+i_L
%\end{align}
%
%}
%
%\frame{
%Define $R_{eq}=(R_s^{-1}+R^{-1})^{-1}$, we solve the differential equation by variable separation:
%\begin{align}\nn
%&I_s=L(R^{-1}_s+R^{-1})\frac{di_L(t)}{dt}+i_L\\\nn
%&\frac{di_L(t)}{dt}=\frac{R_{eq}(I_s-i_L)}{L}\Rightarrow \frac{R_{eq}di_L(t)}{L(I_s-i_L)}=dt\\\nn
%&\int_{i_L(t_0)}^{i_L(t)}\frac{dx}{(I_s-x)}=\frac{R_{eq}}{L}\int_{t_0}^{t}dy
%\end{align}
%
%\begin{exampleblock}{Current through the Inductor}
%\begin{align}\nn
%i_L(t)=I_s\left(1-\text{e}^{-\frac{R_{eq}}{L}t}\right)
%\end{align}
%\end{exampleblock}
%
%}
%
%
%
%\frame{
%\begin{exampleblock}{Voltage in the inductor}
%\begin{align}\nn
%V_L(t)&=L\frac{di_L(t)}{dt}=I_sR_{eq}\text{e}^{-\frac{R_{eq}}{L}t}
%\end{align}
%\end{exampleblock}
%\begin{block}{Current through resistors $R$ and $R_s$}
%\begin{align}\nn
%i_R(t)=\frac{V_L(t)}{R}=I_s\frac{R_{eq}}{R}\text{e}^{-\frac{R_{eq}}{L}t}\\\nn
%i_{R_s}(t)=\frac{V_L(t)}{R_s}=I_s\frac{R_{eq}}{R}\text{e}^{-\frac{R_{eq}}{L}t}
%\end{align}
%\end{block}
%
%}
%
%\section{General solution to a first order linear differential equation}
%
%\frame{
%Compute $x(t)$ such that $x(t_0)=x_0$ if 
%\begin{align}\nn
%A\frac{dx(t)}{dt}+Bx(t)=C
%\end{align}
%
%
%\begin{exampleblock}{Sol.}
%\begin{align}\nn
%x(t)=\frac{C}{B}\left(1-\text{e}^{-\frac{B}{A}(t-t_0)}\right)+x_0\text{e}^{-\frac{B}{A}(t-t_0)}
%\end{align}
%\end{exampleblock}
%}

\end{document}
