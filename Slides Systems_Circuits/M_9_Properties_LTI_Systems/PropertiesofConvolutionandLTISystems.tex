\documentclass[10pt,makeidx]{beamer}

\mode<presentation>
{
 \usetheme{Boadilla}
\pagestyle{empty}

\setbeamerfont*{frametitle}{size=\normalsize,series=\bfseries}
\setbeamerfont*{block}{size=\normalsize,series=\bfseries}
%\setbeamertemplate{blocks}[rounded][shadow=true]
}

\definecolor{links}{HTML}{2A1B81}
\hypersetup{colorlinks,linkcolor=,urlcolor=links}


\usepackage{etex}
% \usepackage{helvet}
\usepackage{amsmath, amssymb}
\usepackage{color}
\usepackage{asymptote}
\usepackage{mathrsfs}
\usepackage{dsfont}
\usepackage{makeidx}
\usepackage{multido}

\usepackage{pst-sigsys,pst-plot,pstricks-add}
\usepackage{pst-pdf}


\definecolor{links}{HTML}{2A1B81}
\hypersetup{colorlinks,linkcolor=,urlcolor=links}

\def\nn{\nonumber}

\definecolor{links}{HTML}{2A1B81}
\hypersetup{colorlinks,linkcolor=,urlcolor=links}

\newcommand{\fs}[2]{#2}




\title[]{Properties of LTI Systems}
\author[\textcolor{blue}{Systems and Circuits}]{\textcolor{darkblue}{Pablo M. Olmos} (olmos@tsc.uc3m.es)\\ \textcolor{darkblue}{Emilio Parrado} (emipar@tsc.uc3m.es)}
\institute{\textcolor{white}{UC3M}}

\definecolor{darkblue}{rgb}{0.0, 0.0, 0.40}
\setbeamercolor{title}{fg=darkblue}
\setbeamercolor{frametitle}{fg=darkblue}
\definecolor{darkgreen}{rgb}{0.0, 0.4, 0.0}


\AtBeginSection[]
{
  \begin{frame}<beamer>{Index}
    \tableofcontents[currentsection,currentsubsection]
  \end{frame}
}

\begin{document}
\frame{
\titlepage
\thispagestyle{empty}
\begin{center}
\includegraphics[scale=0.05]{Figures/uc3m-logo.pdf}
\end{center}
}

\section{Properties of LTI Systems}

\frame{

\begin{itemize}
\item LTI systems are completely determined by their impulse response $h(t)$ ($h[n]$).
\end{itemize}
\begin{itemize}
\item Consequently, we should be able to deduce further properties of these systems just by inspecting  $h(t)$ ($h[n]$):
\begin{enumerate}
\item Memory.
\item Invertibility.
\item Causality.
\item Stability.
\end{enumerate}
\end{itemize}

%\begin{alertblock}{ }
%We don't need the LIT system input/output relationship
%\end{alertblock}

}

\frame{
\frametitle{Memoryless  discrete-time LTI systems}

%\begin{exampleblock}{}
%A system $x[n]\rightarrow y[n]$ is said to be memoryless if the output a one time instant $n_0$, $y[n_0]$, only depends on the input at the same time instant, i.e. $x[n_0]$.
%\end{exampleblock}

Given the impulse response of a LTI system $h[n]$ and its output for an arbitrary input $x[n]$
\begin{align}\nn
y[n]=\sum_{k=-\infty}^{\infty}x[k]h[n-k]=x[n]h[0]+\sum_{\substack{k=-\infty\\k\neq n}}^{\infty}x[k]h[n-k],
\end{align}
we observe that the system is memoryless only if $h[n]=0$ for any $n\neq0$. 

\begin{block}{Memoryless discrete-time LTI systems}
The impulse response is of the form
\begin{align}\nn
h[n]=a\delta[n]
\end{align}
where $a\in\mathbb{C}$ is a constant.
\end{block}


}

\frame{
\frametitle{Memoryless  continuous-time LTI systems}

\begin{exampleblock}{}
Given the impulse response of an LTI system $h(t)$ and its output for an arbitrary input $x(t)$
\begin{align*}
y(t)=\int_{-\infty}^{\infty}{x(\tau)h(t-\tau)d\tau}=\int_{-\infty}^{\infty}{x(t-\nu)h(\nu)d\nu},
\end{align*}
we conclude that the system is memoryless only if $h(t)=a\delta(t)$, where $a\in\mathbb{C}$.
\end{exampleblock}

Hint: $h(t)$ must be zero at any point except at $t=0$ to remove the dependency  of $y(t)$ on $x(t')$ at any $t'\neq t$. Besides, because of the differential $d\nu$ inside the integral, $h(0)$ must tend to $\infty$. Indeed, $h(t)$ must be of the form $a\delta(t)$.


}
%\frame{
%
%\begin{exampleblock}{For any $t_0=k_0\Delta$}
%\begin{align*}
%y_{\Delta}(k_0\Delta)=x(k_0\Delta)h_\Delta(0)\Delta+\sum_{\substack{k=-\infty\\k\neq k_0}}^{\infty}{x(k\Delta) {\color{blue}h_\Delta(k_0\Delta-k\Delta)}\Delta}
%\end{align*}
%if $h_\Delta(t)=0$ for $t\neq0$, then $y_{\Delta}(k_0\Delta)$ only depends on $x(k_0\Delta)$!
%\end{exampleblock}
%
%Besides, in the limit $\Delta\rightarrow 0$,  we need
%\begin{align*}
%\lim_{\Delta\rightarrow 0} h_\Delta(0)\Delta =a,
%\end{align*}
%where $a\in\mathbb{C}$ is a constant.  Otherwise,  $\lim_{\Delta\rightarrow 0}y_{\Delta}(t_0)=0$ !!

%\begin{alertblock}{}
%\begin{align*}
%\lim_{\Delta\rightarrow 0} \Delta  h_\Delta(t) =
%\left\{\begin{array}{cc}
%a & t=0\\
%0 & t\neq 0
%\end{array}
%\right.
%=a\delta (t)
%\end{align*}
%\end{alertblock}
%}

\frame{
\frametitle{Causality in LTI Systems}
\begin{exampleblock}{}
A system is causal if the output at any time depends only on the  input at the present time and at any time in the past. 
\end{exampleblock}

Given two LTI systems with impulse response $h[n]$ and $h(t)$ and the output for an arbitrary input  $x[n]$ and $x(t)$
\begin{align}\nn
y[n]=\sum_{k=-\infty}^{\infty}x[k]h[n-k], \qquad y(t)=\int_{-\infty}^{\infty}{x(\tau)h(t-\tau)d\tau},
\end{align}
%the system is causal if $y[n]$ only depends on $x[n]$, $x[n-1]$, $\ldots$
the systems are causal if 
\begin{block}{}
\begin{align}\nn
h[n]=0 \quad n<0,\\\nn
h(t)=0 \quad t<0.
\end{align}
\end{block}
}


\frame{
\frametitle{Stability in LTI systems}
%\begin{exampleblock}{Bounded Input, Bounded Output (BIBO) stability}
%If a system is BIBO stable, then the output will be bounded for every input to the system that is bounded. 
%%
%%\\
%%A signal is bounded if there is a finite value $B$ such that the signal never exceeds $B$, that is
%%\begin{align}\nn
%%|x[n]|&\leq B\qquad \forall n\in\mathbb{Z} \\\nn
%%|x(t)|&\leq B\qquad \forall t\in\mathbb{R}
%%\end{align}
%\end{exampleblock}

Given a discrete-time LTI system $h[n]$ and an input signal $x[n]$ such that $|x[n]|<B$ $\forall n$, the output signal
\begin{align}\nn
y[n]=\sum_{k=-\infty}^{\infty}x[k]h[n-k].
\end{align}
can be bounded as follows
\begin{align}\nn
|y[n]|&=|\sum_{k=-\infty}^{\infty}x[k]h[n-k]|\leq \sum_{k=-\infty}^{\infty} |x[k]||h[n-k]|\\\nn
&<\sum_{k=-\infty}^{\infty}  B|h[n-k]|=B\sum_{k=-\infty}^{\infty} |h[n-k]|.
\end{align}

\begin{alertblock}{}
The LTI system is stable if $\displaystyle \sum_{k=-\infty}^{\infty} |h[k]|<\infty$
\end{alertblock}
}

\frame{

\begin{itemize}
\item For continuous-time LTI systems the stability condition can be proven in a similar way (\textbf{Exercise}).
\end{itemize}



\begin{block}{BIBO stable LTI systems}
A discrete-time LTI system with impulse response $h[n]$ is stable if
\begin{align}\nn
\sum_{k=-\infty}^{\infty} |h[k]|<\infty
%,\\\nn
%\int_{-\infty}^{\infty}|h(\tau)|\text{d}\tau<\infty.
\end{align}
A continuous-time LTI system with impulse response $h(t)$ is stable if
\begin{align*}
\int_{-\infty}^{\infty}|h(\tau)|\text{d}\tau<\infty.
\end{align*}
\end{block}

}

\frame{
\frametitle{Invertibility of LTI Systems}

\begin{exampleblock}{}
\begin{itemize}
\item A system is said to be invertible if distinct inputs lead to distinct outputs.
\item We have a one to one input-output correspondence.
\item By observing its output, we can determine its input. 
\end{itemize}
\end{exampleblock}

A LTI system $h(t)$ is invertible if we are able to find another LTI system with impulse response $h_{\text{inv}}(t)$ such that:
\vspace{0.5cm}

\begin{center}
\begin{pspicture}[](8,2)
   \pssignal(0,1){x}{$x(t)$}
   \psblock(2,1){a}{$h(t)$}
   \psblock(5,1){b}{$h_{\text{inv}}(t)$}
   \pssignal(8,1){y}{$x(t)$}
   %-----------------
   \psset{arrows=->}
   \ncline{x}{a}  \ncline{a}{b}  \ncline{b}{y}
\end{pspicture}
\end{center}

}

\frame{
By the associative property
\begin{center}
\begin{pspicture}[](7,2)
   \pssignal(0,1){x}{$x(t)$}
   \psblock(3,1){a}{$h(t)\ast h_{\text{inv}}(t)$}
   %\psblock(4,1){b}{$h[n], H(z)$}
   \pssignal(6,1){y}{$x(t)$}
   %-----------------
   \psset{arrows=->}
   \ncline{x}{a}  \ncline{a}{y}  %\ncline{b}{y}
\end{pspicture}
\vspace{0.5cm}
\end{center}

Therefore,
\begin{align}\nn
h(t)\ast h_{\text{inv}}(t)=\delta(t)
\end{align}

\begin{block}{Invertible LTI systems}
\begin{itemize}
\item $h[n]$ is invertible $\Longleftrightarrow$ there exists $h_{\text{inv}}[n]$ such that $h[n]\ast h_{\text{inv}}[n]=\delta[n]$.
\item  $h(t)$ is invertible $\Longleftrightarrow$ there exists $h_{\text{inv}}(t)$ such that $h(t)\ast h_{\text{inv}}(t)=\delta(t)$.
\end{itemize}
\end{block}

}

\section{The unit step response of an LTI systems}

\frame{
\frametitle{The unit step response of an LTI systems}

\begin{itemize}
\item LTI systems are completely characterized by their impulse response.
%We have seen that the representation of an LTI system in terms of its unit impulse response allow us to obtain explicit characterization of the system properties.
\item Sometimes, it is easier to observe the system's output response to the unit step.
\item Let's see how we can obtain the impulse response from the system's output response to the unit step.
\end{itemize}
}

\frame{
Assume an LTI system for which $h[n]$ is unknown:
\begin{center}
\begin{pspicture}[](5,2)
   \pssignal(0,1){x}{$u[n]$}
   \psblock(2,1){a}{$h[n]$}
   %\psblock(4,1){b}{$h[n], H(z)$}
   \pssignal(4,1){y}{$s[n]$}
   %-----------------
   \psset{arrows=->}
   \ncline{x}{a}  \ncline{a}{y}  %\ncline{b}{y}
\end{pspicture}
\vspace{0.5cm}
\end{center}

\begin{align}\nn
s[n]&=u[n]\ast h[n]=\sum_{k=-\infty}^{\infty}u[k]h[n-k]=\sum_{k=0}^{\infty}h[n-k]=\sum_{v=-\infty}^{n}h[v]
\end{align}

Therefore,

\begin{exampleblock}{}
\begin{align}\nn
h[n]=s[n]-s[n-1]
\end{align}
\end{exampleblock}

}

\frame{

Assume a LTI system for which $h(t)$ is unknown:
\begin{center}
\begin{pspicture}[](5,2)
   \pssignal(0,1){x}{$u(t)$}
   \psblock(2,1){a}{$h(t)$}
   %\psblock(4,1){b}{$h[n], H(z)$}
   \pssignal(4,1){y}{$s(t)$}
   %-----------------
   \psset{arrows=->}
   \ncline{x}{a}  \ncline{a}{y}  %\ncline{b}{y}
\end{pspicture}
\vspace{0.5cm}
\end{center}

\begin{align}\nn
s(t)&=u(t)\ast h(t)=\int_{-\infty}^{\infty} u(\tau)h(t-\tau) d\tau=\int_{0}^{\infty}h(t-\tau) d\tau=\int_{-\infty}^{t}h(\psi) d\psi
\end{align}

Therefore,

\begin{exampleblock}{}
\begin{align}\nn
h(t)=\frac{\partial s(t)}{\partial t}
\end{align}
\end{exampleblock}


}

%
%
%\frame{
%\frametitle{Time Invariance}
%
%\begin{block}{}
%A system is time-invariant if a time shift in the input signal causes a time shift in the output signal.
%\end{block}
%\vspace{0.5cm}
%
%\begin{itemize}
%\item If $y[n]=f(x[n])$, the system is invariant if $f(x[n-n_0])=y[n-n_0]$. 
%\item If $y(t)=f(x(t))$, the system is invariant if $f(x(t-t_0))=y(t-t_0)$. 
%\end{itemize}
%}
%
%
%\frame{
%\frametitle{Linearity}
%Linear system posses the important property of superposition.
%
%\begin{exampleblock}{}
% For any system, consider two arbitrary inputs and their respective outputs:
%\begin{align}\nn
%x_1(t)\rightarrow y_1(t)\\\nn
%x_2(t)\rightarrow y_2(t),
%\end{align}
%the system is linear if 
%\begin{align}\nn
%ax_1(t)+bx_2(t)\rightarrow ay_1(t)+by_2(t)
%\end{align}
%for any two complex constant $a,b\in\mathbb{C}$.
%\end{exampleblock}
%
%\begin{block}{Linear discrete-time signals}
%\begin{align}\nn
%ax_1[n]+bx_2[n]\rightarrow ay_1[n]+by_2[n]
%\end{align}
%\end{block}
%
%}
%
%
%\section{Discrete-time LTI systems}
%
%\frame{
%
%\begin{pspicture}[](5,2)
%   \pssignal(0,1){x}{$x[n]$}
%   \psblock(2,1){a}{LTI}
%   %\psblock(4,1){b}{$h[n], H(z)$}
%   \pssignal(4,1){y}{$y[n]$?}
%   %-----------------
%   \psset{arrows=->}
%   \ncline{x}{a}  \ncline{a}{y}  %\ncline{b}{y}
%\end{pspicture}
%
%\vspace{0.5cm}
%Remember that any discrete-time sequence $x[n]$ can be decomposed as a linear combination of unit impulses:
%\begin{block}{}
%\begin{align}\nn
%x[n]&=\sum_{k=-\infty}^{\infty}x[k]\delta[n-k]\\\nn
%&=\ldots+x[-20]\delta[n+20]+x[-19]\delta[n+19]+\ldots\\\nn
%&+x[-1]\delta[n+1]+x[0]\delta[n]+x[1]\delta[n-1]+\ldots+x[k]\delta[n-k]
%\end{align}
%\end{block}
%}
%
%\frame{
%If we are able to compute
%\vspace{0.5cm}
%
%\begin{center}
%\begin{pspicture}[](6.5,5)
%	
%   \pssignal(0,5){x1}{$\delta[n]$}
%    \psblock(2,5){a1}{LTI}
%    \pssignal(4,5){y1}{$h_0[n]$}   
%   \pssignal(0,4){x2}{$\delta[n+1]$}
%    \psblock(2,4){a2}{LTI}
%    \pssignal(4,4){y2}{$h_{-1}[n]$}      
%     \pssignal(0,3){x3}{$\delta[n-1]$}
%    \psblock(2,3){a3}{LTI}
%    \pssignal(4,3){y3}{$h_{1}[n]$}     
%    
%    \ldotsnode[angle=90](0,2){dots}
%    \ldotsnode[angle=90](2,2){dots}
%    \ldotsnode[angle=90](4,2){dots}
%    
%    \pssignal(0,1){x4}{$\delta[n-k]$}
%    \psblock(2,1){a4}{LTI}
%    \pssignal(4,1){y4}{$h_{k}[n]$}   
%    
%   \psset{arrows=->}
%   \ncline{x1}{a1}  \ncline{a1}{y1}    
%   \ncline{x2}{a2}  \ncline{a2}{y2}  
%    \ncline{x3}{a3}  \ncline{a3}{y3} 
%     \ncline{x4}{a4}  \ncline{a4}{y4}  
%       
%\end{pspicture}
%\end{center}
%
%Since the system is linear, we can make use of the superposition property.
%}
%
%\frame{
%
%
%\begin{pspicture}[](9,2)
%   \pssignal(0,1){x}{$x[n]=\sum_{k=-\infty}^{\infty}x[k]\delta[n-k]$}
%   \psblock(4,1){a}{LTI}
%   %\psblock(4,1){b}{$h[n], H(z)$}
%   \pssignal(6,1){y}{$y[n]$}
%   %-----------------
%   \psset{arrows=->}
%   \ncline{x}{a}  \ncline{a}{y}  %\ncline{b}{y}
%\end{pspicture}
%
%\vspace{0.5cm}
%\begin{exampleblock}{}
%\begin{align}\nn
%y[n]=\sum_{k=-\infty}^{\infty}x[k]h_{k}[n]
%\end{align}
%\end{exampleblock}
%
%\vspace{0.5cm}
%Besides, the system is time-invariant!
%}
%
%\frame{
%
%\begin{center}
%\begin{pspicture}[](8,5)
%	
%   \pssignal(0,5){x1}{$\delta[n]$}
%    \psblock(2,5){a1}{LTI}
%    \pssignal(5,5){y1}{$h_0[n]=h[n]$}   
%   \pssignal(0,4){x2}{$\delta[n+1]$}
%    \psblock(2,4){a2}{LTI}
%    \pssignal(5,4){y2}{$h_{-1}[n]=h[n+1]$}      
%     \pssignal(0,3){x3}{$\delta[n-1]$}
%    \psblock(2,3){a3}{LTI}
%    \pssignal(5,3){y3}{$h_{1}[n]=h[n-1]$}     
%    
%    \ldotsnode[angle=90](0,2){dots}
%    \ldotsnode[angle=90](2,2){dots}
%    \ldotsnode[angle=90](4,2){dots}
%    
%    \pssignal(0,1){x4}{$\delta[n-k]$}
%    \psblock(2,1){a4}{LTI}
%    \pssignal(5,1){y4}{$h_{k}[n]=h[n-k]$}   
%    
%   \psset{arrows=->}
%   \ncline{x1}{a1}  \ncline{a1}{y1}    
%   \ncline{x2}{a2}  \ncline{a2}{y2}  
%    \ncline{x3}{a3}  \ncline{a3}{y3} 
%     \ncline{x4}{a4}  \ncline{a4}{y4}  
%       
%\end{pspicture}
%\end{center}
%
%Therefore, for any input $x[n]$, if $h[n]$ is known then the system output is given by
%
%\begin{alertblock}{}
%\begin{align}\nn
%y[n]=\sum_{k=-\infty}^{\infty}x[k]h_{k}[n]=\sum_{k=-\infty}^{\infty}x[k]h[n-k]
%\end{align}
%\end{alertblock}
%
%}
%
%\frame{
%
%\begin{itemize}
%\item $h[n]$ is the system's impulse response.
%\item Any LTI system is completely defined by $h[n]$:
%\vspace{0.3cm}
%\begin{center}
%\begin{pspicture}[](5,2)
%   \pssignal(0,1){x}{$x[n]$}
%   \psblock(2,1){a}{$h[n]$}
%   %\psblock(4,1){b}{$h[n], H(z)$}
%   \pssignal(4,1){y}{$y[n]$}
%   %-----------------
%   \psset{arrows=->}
%   \ncline{x}{a}  \ncline{a}{y}  %\ncline{b}{y}
%\end{pspicture}
%\end{center}
%\end{itemize}
%
%
%\begin{block}{Convolution}
%Given two discrete-time signals $x[n]$ and $h[n]$:
%\begin{align}\nn
%\sum_{k=-\infty}^{\infty}x[k]h[n-k]=x[n]\ast h[n]
%\end{align}
%is the convolution operation between them.
%\end{block}
%
%
%}
%
%
%\frame{
%\frametitle{Example I}
%
%\psset{xunit=0.75cm,yunit=0.75cm}
%
%\begin{tabular}{|c|c|}
%\hline
%\begin{pspicture}[showgrid](-2,-1)(4,3)
%  \rput(0,0){\psaxeslabels(0,0)(-2,0)(4,0){$n$}{}
%     \rput[tl](-2,3){$x[n]$}
%     \psstem[style=Stem,linecolor=blue](-1,1)
%            {0,1,2,0}}
%\end{pspicture}
%&
%\begin{pspicture}[showgrid](-1,-1)(5,3)
%  \rput(0,0){\psaxeslabels(0,0)(-1,0)(5,0){$n$}{}
%     \rput[tl](-1,3){$h[n]$}
%     \psstem[style=Stem,linecolor=blue](0,1)
%            {0,1,1,1,0}}
%\end{pspicture}\\
%\hline
%\end{tabular}
%
%
%}
%
%\frame{
%\frametitle{Example I}
%
%\psset{xunit=0.75cm,yunit=0.75cm}
%
%\begin{center}
%
%\begin{pspicture}[showgrid](0,-1)(6,3)
%  \rput(0,0){\psaxeslabels(0,0)(0,0)(6,0){$n$}{}
%     \rput[tl](0,3){$y[n]$}
%     \psstem[style=Stem,linecolor=red](1,1)
%            {1,3,3,2,0}}
%\end{pspicture}
%
%\end{center}
%
%}
%
%\frame{
%\frametitle{Example II}
%
%\psset{xunit=0.75cm,yunit=0.75cm}
%
%\begin{center}
%
%\begin{tabular}{|c|c|}
%\hline
%\begin{pspicture}[showgrid](-1,-1)(4,5)
%  \rput(0,0){\psaxeslabels(0,0)(-1,0)(4,0){$n$}{}
%     \rput[tl](-1,3){$x[n]$}
%     \psstem[style=Stem,linecolor=blue](-1,1)
%            {0,4,1,2,5}}
%\end{pspicture}
%&
%\begin{pspicture}[showgrid](-2,-1)(3,3)
%  \rput(0,0){\psaxeslabels(0,0)(-2,0)(3,0){$n$}{}
%     \rput[tl](-2,3){$h[n]$}
%     \psstem[style=Stem,linecolor=blue](-2,1)
%            {0,1,2,-1}}
%\end{pspicture}\\
%\hline
%\end{tabular}
%
%\end{center}
%
%\url{http://www.youtube.com/watch?v=yyTu0SXeW1M}
%
%}
%
%\frame{
%\frametitle{Example II}
%
%\psset{xunit=0.5cm,yunit=0.5cm}
%
%\begin{center}
%
%\begin{pspicture}[showgrid](-2,-5)(7,10)
%  \rput(0,0){\psaxeslabels(0,0)(-2,0)(7,0){$n$}{}
%     \rput[tl](-2,10){$y[n]$}
%     \psstem[style=Stem,linecolor=red](-2,1)
%            {0,4,9,0,8,8,-5,0}}
%\end{pspicture}
%
%\end{center}
%
%}
%
%
%
%\section{Continuous-time LTI systems}
%
%\frame{
%
%\textbf{\color{red}{The discussion for continuous-time LTI systems is just a generalization of the discrete-time case.}}
%
%
%\begin{center}
%\begin{pspicture}[](5,2)
%   \pssignal(0,1){x}{$x(t)$}
%   \psblock(2,1){a}{LTI}
%   %\psblock(4,1){b}{$h[n], H(z)$}
%   \pssignal(4,1){y}{$y(t)?$}
%   %-----------------
%   \psset{arrows=->}
%   \ncline{x}{a}  \ncline{a}{y}  %\ncline{b}{y}
%\end{pspicture}
%\end{center}
%
%\vspace{0.5cm}
%Remember that any continuous-time signal $x(t)$ can be decomposed as a linear combination of an infinite number of
%impulses:
%\begin{block}{}
%\begin{align}\nn
%x(t)=\int_{-\infty}^{\infty}x(\tau)\delta(t-\tau)d\tau
%\end{align}
%\end{block}
%}
%
%
%\frame{
%
%If the system's output is known for $\delta(t-\tau)$ for any $\tau\in\mathbb{R}$:
%
%\begin{center}
%\begin{pspicture}[](5,2)
%   \pssignal(0,1){x}{$\delta(t-\tau)$}
%   \psblock(2,1){a}{LTI}
%   %\psblock(4,1){b}{$h[n], H(z)$}
%   \pssignal(4,1){y}{$h_{\tau}(t)$}
%   %-----------------
%   \psset{arrows=->}
%   \ncline{x}{a}  \ncline{a}{y}  %\ncline{b}{y}
%\end{pspicture}
%\end{center}
%
%Then, by the superposition property,
%
%\begin{center}
%\begin{pspicture}[](7,2)
%   \pssignal(0,1){x}{$\displaystyle x(t)=\int_{-\infty}^{\infty}x(\tau)\delta(t-\tau)d\tau$}
%   \psblock(4,1){a}{LTI}
%   %\psblock(4,1){b}{$h[n], H(z)$}
%   \pssignal(6,1){y}{$y(t)$}
%   %-----------------
%   \psset{arrows=->}
%   \ncline{x}{a}  \ncline{a}{y}  %\ncline{b}{y}
%\end{pspicture}
%\end{center}
%
%\begin{exampleblock}{}
%\begin{align}\nn
%y(t)=\int_{-\infty}^{\infty} x(\tau)h_{\tau}(t) d\tau 
%\end{align}
%\end{exampleblock}
%
%}
%
%\frame{
%
%If the system is time-invariant:
%
%\begin{center}
%\begin{pspicture}[](7,2)
%   \pssignal(0,2){x}{$\delta(t)$}
%   \psblock(2,2){a}{LTI}
%   %\psblock(4,1){b}{$h[n], H(z)$}
%   \pssignal(5,2){y}{$h_{0}(t)=h(t)$}
%   
%    \pssignal(0,1){x1}{$\delta(t-\tau)$}
%   \psblock(2,1){a1}{LTI}
%   %\psblock(4,1){b}{$h[n], H(z)$}
%   \pssignal(5,1){y1}{$h_{\tau}(t)=h(t-\tau)$}  
%   %-----------------
%   \psset{arrows=->}
%   \ncline{x}{a}  \ncline{a}{y}  %\ncline{b}{y}
%   \ncline{x1}{a1}  \ncline{a1}{y1}
%\end{pspicture}
%\end{center}
%
%then, we finally obtain
%
%\begin{center}
%\begin{pspicture}[](7,2)
%   \pssignal(0,1){x}{$\displaystyle x(t)=\int_{-\infty}^{\infty}x(\tau)\delta(t-\tau)d\tau$}
%   \psblock(4,1){a}{LTI}
%   %\psblock(4,1){b}{$h[n], H(z)$}
%   \pssignal(6,1){y}{$y(t)$}
%   %-----------------
%   \psset{arrows=->}
%   \ncline{x}{a}  \ncline{a}{y}  %\ncline{b}{y}
%\end{pspicture}
%\end{center}
%
%\begin{exampleblock}{}
%\begin{align}\nn
%y(t)=\int_{-\infty}^{\infty} x(\tau)h(t-\tau) d\tau 
%\end{align}
%\end{exampleblock}
%
%}
%
%\frame{
%
%\begin{itemize}
%\item $h(t)$ is the system's response to the impulse $\delta(t)$.
%\item Any LTI system is completely defined by $h(t)$:
%\vspace{0.3cm}
%\begin{center}
%\begin{pspicture}[](5,2)
%   \pssignal(0,1){x}{$x(t)$}
%   \psblock(2,1){a}{$h(t)$}
%   %\psblock(4,1){b}{$h[n], H(z)$}
%   \pssignal(4,1){y}{$y(t)$}
%   %-----------------
%   \psset{arrows=->}
%   \ncline{x}{a}  \ncline{a}{y}  %\ncline{b}{y}
%\end{pspicture}
%\end{center}
%\end{itemize}
%
%
%\begin{block}{Convolution}
%Given two signals $x(t)$ and $h(t)$:
%\begin{align}\nn
%\int_{-\infty}^{\infty} x(\tau)h(t-\tau) d\tau=x(t)\ast h(t)
%\end{align}
%is the continuous-time convolution operation between them.
%\end{block}
%}
%
%\frame{
%
%\frametitle{Example}
%
%\begin{center}
%\begin{pspicture}[](5,2)
%   \pssignal(0,1){x}{$x(t)$}
%   \psblock(2,1){a}{$h(t)=u(t)$}
%   %\psblock(4,1){b}{$h[n], H(z)$}
%   \pssignal(4,1){y}{$y(t)?$}
%   %-----------------
%   \psset{arrows=->}
%   \ncline{x}{a}  \ncline{a}{y}  %\ncline{b}{y}
%\end{pspicture}
%\end{center}
%
%\begin{align}\nn
%x(t)=\left\{
%\begin{array}{cc}
%t & 0\leq t\leq 1\\
%1 & 1<t\leq 2\\
%0 & \text{otherwise}
%\end{array}
%\right.
%\end{align}
%
%\textbf{Sol.}:
%
%\begin{align}\nn
%y(t)=\left\{
%\begin{array}{cc}
%0 & \leq t\leq0\\\\
%\frac{t^2}{2} & 0< t\leq 1\\\\
%\frac{1}{2}+(t-1) & 1<t\leq 2\\\\
%\frac{3}{2} & t>2
%\end{array}
%\right.
%\end{align}
%
%}


%
%\section{Continuous-time and discrete-time systems}
%
%\frame{
%
%\begin{center}
%\begin{tabular}{c}
%\begin{pspicture}[](6,2)
%   \pssignal(0,1){x}{$x(t)$}
%   \psblock(3,1){a}{Continuous-time System}
%   %\psblock(4,1){b}{$h[n], H(z)$}
%   \pssignal(6,1){y}{$y(t)$}
%   %-----------------
%   \psset{arrows=->}
%   \ncline{x}{a}  \ncline{a}{y}  %\ncline{b}{y}
%\end{pspicture}
%\\\\\\
%\begin{pspicture}[](6,2)
%   \pssignal(0,1){x}{$x[n]$}
%   \psblock(3,1){a}{Discrete-time System}
%   %\psblock(4,1){b}{$h[n], H(z)$}
%   \pssignal(6,1){y}{$y[n]$}
%   %-----------------
%   \psset{arrows=->}
%   \ncline{x}{a}  \ncline{a}{y}  %\ncline{b}{y}
%\end{pspicture}
%\end{tabular}
%\end{center}
%
%Symbolically, we represent a system as:
%\begin{align}\nn
%x(t)\rightarrow y(t)\\\nn
%x[n]\rightarrow y[n]
%\end{align}
%
%}
%
%\frame{
%\frametitle{Interconnection of systems}
%
%Any complex system is designed as the interconnection of simpler systems:
%
%
%\begin{center}
%\begin{tabular}{c}
%\textbf{Cascade interconnection}\\\\
%\begin{pspicture}[](8,2)
%   \pssignal(0,1){x}{Input}
%   \psblock(2,1){a}{System 1}
%   \psblock(5,1){b}{System 2}
%   \pssignal(8,1){y}{Output}
%   %-----------------
%   \psset{arrows=->}
%   \ncline{x}{a}  \ncline{a}{b}  \ncline{b}{y}
%\end{pspicture}
%\\\\
%\textbf{Parallel interconnection}\\\\
%\begin{pspicture}[](6.5,2)
%	
%   \pssignal(0,1){x}{Input}
%   \dotnode(1,1){b}
%   \dotnode(1,2){c}
%   \dotnode(1,0){d}
%   \psblock(3,2){e}{System 1}
%   \psblock(3,0){f}{System 2}
%   \dotnode(5,2){g}
%   \dotnode(5,0){h}
%   \pscircleop(5,1){oplus}
%   \pssignal(6.25,1){i}{$y[n]$}
%   \ncline{x}{b}  \ncline{b}{c}  \ncline{b}{d}
%    \ncline{e}{g} \ncline{f}{h}
%   \psset{style=RoundCorners ,style=Arrow}
%     \ncline{c}{e} \ncline{d}{f}
%   \ncline{g}{oplus}  \ncline{h}{oplus} \ncline{oplus}{i}
%\end{pspicture}
%\end{tabular}
%\end{center}
%
%}
%
%\frame{
%\frametitle{Interconnection of systems}
%
%\begin{center}
%\begin{tabular}{c}
%\textbf{Series/Parallel interconnection}\\\\
%\begin{pspicture}[](8,2)
%	
%   \pssignal(0,1){x}{Input}
%   \dotnode(1,1){b}
%   \dotnode(1,2){c}
%   \dotnode(1,0){d}
%   \psblock(2.5,2){e}{System 1}
%   \psblock(5,2){j}{System 2}
%   \psblock(4,0){f}{System 3}
%   \dotnode(6.5,2){g}
%   \dotnode(6.5,0){h}
%   \pscircleop(6.5,1){oplus}
%   \pssignal(8,1){i}{$y[n]$}
%   
%   %Conexiones
%   \ncline{x}{b}  \ncline{b}{c}  \ncline{b}{d}
%     \ncline{f}{h} \ncline{j}{g}
%   \psset{style=RoundCorners ,style=Arrow}
%     \ncline{c}{e} \ncline{d}{f}
%   \ncline{g}{oplus}  \ncline{h}{oplus} \ncline{oplus}{i}
%   \ncline{e}{j}
%\end{pspicture}
%\end{tabular}
%\end{center}
%
%\begin{block}{}
%In general, the order is NOT interchangeable.
%\end{block}
%
%}
%
%\frame{
%\begin{align}\nn
%y[n]=(2x[n]-x[n]^2)^2
%\end{align}
%\vspace{1cm}
%
%\begin{center}
%\begin{pspicture}[](7.5,2)
%	
%   \pssignal(0,1){x}{$x[n]$}
%   \dotnode(1,1){b}
%   \dotnode(1,2){c}
%   \dotnode(1,0){d}
%   \psblock(3,2){e}{Multiply by 2}
%   \psblock(3,0){f}{Square}
%   \dotnode(5,2){g}
%   \dotnode(5,0){h}
%   \pscircleop(5,1){oplus}
%   \psblock(6.25,1){j}{Square}
%   \pssignal(8.25,1){i}{$y[n]$}
%   \ncline{x}{b}  \ncline{b}{c}  \ncline{b}{d}
%    \ncline{e}{g} \ncline{f}{h}
%   \psset{style=RoundCorners ,style=Arrow}
%     \ncline{c}{e} \ncline{d}{f}
%      \ncline{oplus}{j} \ncline{j}{i}
%   \ncstar{->}{ncline}[naput]{h -}{oplus} %\ncline{h}{oplus}
%   \ncline{g}{oplus}
%\end{pspicture}
%\end{center}
%
%}
%
%\section{Properties of systems}
%
%\frame{
%\frametitle{Systems with and without memory}
%
%\begin{exampleblock}{}
%A system $x(t)\rightarrow y(t)$ is said to be memoryless if the output a one time instant $t_0$, $y(t_0)$, only depends on the input at the same time instant, i.e. $x(t_0)$.
%\end{exampleblock}
%
%\textbf{Examples:}
%
%\begin{itemize}
%\item $y[n]=(2x[n]-x[n]^2)^2$ is memoryless.
%\item $y(t)=\frac{\partial x(t)}{\partial t}$ is memoryless.
%\item $y(t)=x(t-3)x(t+2)$ is a system with memory.
%\item $y[n]=x(t-1)$ is a system with memory.
%\end{itemize}
%
%}
%
%\frame{
%\frametitle{Invertibility and inverse systems}
%
%\begin{block}{}
%\begin{itemize}
%\item A system is said to be invertible if distinct inputs lead to distinct outputs.
%\item We have a one to one input-output correspondence.
%\item By observing its output, we can determine its input. 
%\end{itemize}
%\end{block}
%
%\begin{tabular}{c}
%\\\\
%\begin{pspicture}[](9,2)
%   \pssignal(0,1){x}{$x(t)$}
%   \psblock(2,1){a}{$y(t)=2x(t)$}
%   \psblock(5.5,1){b}{$z(t)=\frac{1}{2}y(t)$}
%   \pssignal(8.25,1){y}{$x(t)$}
%   %-----------------
%   \psset{arrows=->}
%   \ncline{x}{a}   \ncline{b}{y}
%   \ncstar{->}{ncline}[naput]{a $y(t)$}{b} %\ncline{h}{oplus}
%\end{pspicture}
%\\\\
%\begin{pspicture}[](13,2)
%   \pssignal(0,1){x}{$x[n]$}
%   \psblock(3,1){a}{$y[n]=\sum_{k=-\infty}^{n}x[k]$}
%   \psblock(8,1){b}{$z[n]=y[n]-y[n-1]$}
%   \pssignal(11,1){y}{$x[n]$}
%   %-----------------
%   \psset{arrows=->}
%   \ncline{x}{a}   \ncline{b}{y}
%    \ncstar{->}{ncline}[naput]{a $y[n]$}{b} %\ncline{h}{oplus}
%\end{pspicture}
%\end{tabular}
%}
%
%\frame{
%
%Are invertible the following systems?
%
%\begin{itemize}
%\item $y[n]=x[2n]$
%\item $y(t)=x(2t)$
%\item $y(t)=x^{2}(t)$
%\end{itemize}
%
%}
%
%\frame{
%\frametitle{Causality}
%
%\begin{exampleblock}{}
%\begin{itemize}
%\item A system is causal if the output at any time depends only on values of the input at the present time and in the past. 
%\item A causal system is also referred to as nonanticipative.
%\item $y(t)=f(x(t))$ is causal if $y(t_0)$ only depends on $x(t)$ for $t\leq t_0$.
%\item $y[n]=f(x[n])$ is causal if $y[n_0]$ only depends on $x[n]$ for $n\leq n_0$.
%\item Memoryless systems are always causal.
%\end{itemize}
%\end{exampleblock}
%
%\begin{itemize}
%\item $y[n]=x[n]-x[n+1]$ is not causal.
%\item $y(t)=x(t+1)$ is not causal.
%\end{itemize}
%
%}
%
%\frame{
%\frametitle{Stability}
%
%\begin{exampleblock}{Intuitive notion}
%A system is stable if finite input values lead to responses that do not diverge.
%\end{exampleblock}
%
%\begin{block}{Bounded Input, Bounded Output (BIBO) stability}
%If a system is BIBO stable, then the output will be bounded for every input to the system that is bounded. \\
%
%A signal is bounded if there is a finite value $B$ such that the signal never exceeds $B$, that is
%\begin{align}\nn
%|x[n]|&\leq B\qquad \forall n\in\mathbb{Z} \\\nn
%|x(t)|&\leq B\qquad \forall t\in\mathbb{R}
%\end{align}
%\end{block}
%
%}
%
%\frame{
%
%The systems
%\begin{align}\nn
%y(t)&=\frac{1}{x(t)-3}\\
%y(t)&=\log(x(t)-3)\nn
%\end{align}
%are not stable.
%}
%
%\frame{
%\frametitle{Time Invariance}
%
%\begin{block}{}
%A system is time-invariant if a time shift in the input signal causes a time shift in the output signal.
%\end{block}
%\vspace{0.5cm}
%
%\begin{itemize}
%\item If $y[n]=f(x[n])$, the system is invariant if $f(x[n-n_0])=y[n-n_0]$. 
%\item If $y(t)=f(x(t))$, the system is invariant if $f(x(t-t_0))=y(t-t_0)$. 
%\end{itemize}
%}
%
%\frame{
%
%\begin{exampleblock}{$y(t)=\sin(x(t))$}
%Let $x_1(t)$ be the input:
%\begin{align}\nn
%y_1(t)=\sin(x_1(t)).
%\end{align}
%Define $x_2(t)=x_1(t-t_0)$:
%\begin{align}\nn
%y_2(t)=\sin(x_2(t))=\sin(x_1(t-t_0))=y_1(t-t_0).
%\end{align}
%The system is time-invariant.
%\end{exampleblock}
%}
%
%\frame{
%
%\begin{alertblock}{$y[n]=nx[n]$}
%Let $x_1[n]$ be the input:
%\begin{align}\nn
%y_1[n]=nx_1[n].
%\end{align}
%Define $x_2[n]=x_1[n-n_0]$:
%\begin{align}\nn
%&y_2[n]=nx_2[n]=nx_1[n-n_0]\\\nn
% &y_1[n-n_0]=(n-n_0)x_1[n-n_0].
%\end{align}
%The system is NOT time-invariant.
%\end{alertblock}
%
%}
%
%\frame{
%\frametitle{Linearity}
%Linear system posses the important property of superposition.
%
%\begin{exampleblock}{}
% For any system, consider two arbitrary inputs and their respective outputs:
%\begin{align}\nn
%x_1(t)\rightarrow y_1(t)\\\nn
%x_2(t)\rightarrow y_2(t),
%\end{align}
%the system is linear if 
%\begin{align}\nn
%ax_1(t)+bx_2(t)\rightarrow ay_1(t)+by_2(t)
%\end{align}
%for any two complex constant $a,b\in\mathbb{C}$.
%\end{exampleblock}
%
%\begin{block}{Linear discrete-time signals}
%\begin{align}\nn
%ax_1[n]+bx_2[n]\rightarrow ay_1[n]+by_2[n]
%\end{align}
%\end{block}
%
%}
%
%\frame{
%The systems 
%\begin{align}\nn
%y[n]=(x[n])^{2},\\\nn
%y[n]=\exp(x[n]),
%\end{align}
%are not linear. 
%\vspace{0.5cm}
%
%The system
%\begin{align}\nn
%y[n]=x[n]-x[n-3]+4x[n-8]
%\end{align}
%is linear.
%
%}
%
%
%
%\frame{
%Videos about signals and systems with examples (University of Arizona State):
%
%\begin{exampleblock}{}
%\url{https://sites.google.com/a/asu.edu/signals-and-systems/\#TOC-System-Properties}
%\end{exampleblock}
%
%}


\end{document}
