\documentclass[10pt,makeidx]{beamer}
\mode<presentation>
{
 \usetheme{Boadilla}
\pagestyle{empty}

\setbeamerfont*{frametitle}{size=\normalsize,series=\bfseries}
\setbeamerfont*{block}{size=\normalsize,series=\bfseries}
%\setbeamertemplate{blocks}[rounded][shadow=true]
}

\usepackage{etex}
% \usepackage{helvet}
\usepackage{amsmath, amssymb}
\usepackage{color}
%\usepackage{asymptote}
\usepackage{mathrsfs}
\usepackage{dsfont}
\usepackage{makeidx}
\usepackage{multido}
\usepackage{adjustbox}

\usepackage{pst-sigsys,pst-plot,pstricks-add}
\usepackage{pst-pdf}

\usepackage[american voltages, american currents,siunitx]{circuitikz}
%\sisetup{load=derived} % loading \siemens


\makeatletter
% create the shape
\pgfcircdeclarebipole{}{\ctikzvalof{bipoles/interr/height 2}}{spst}{\ctikzvalof{bipoles/interr/height}}{\ctikzvalof{bipoles/interr/width}}{

    \pgfsetlinewidth{\pgfkeysvalueof{/tikz/circuitikz/bipoles/thickness}\pgfstartlinewidth}

    \pgfpathmoveto{\pgfpoint{\pgf@circ@res@left}{0pt}}
    \pgfpathlineto{\pgfpoint{.6\pgf@circ@res@right}{\pgf@circ@res@up}}
    \pgfusepath{draw}   
}

% make the shape accessible with nice syntax
\def\pgf@circ@spst@path#1{\pgf@circ@bipole@path{spst}{#1}}
\tikzset{switch/.style = {\circuitikzbasekey, /tikz/to path=\pgf@circ@spst@path, l=#1}}
\tikzset{spst/.style = {switch = #1}}
\makeatother



\definecolor{links}{HTML}{2A1B81}
\hypersetup{colorlinks,linkcolor=,urlcolor=links}

\def\nn{\nonumber}
\def\V{\mathbf{V}}
\def\I{\mathbf{I}}
\def\Z{\mathbf{Z}}
\def\e{\text{e}}

\definecolor{links}{HTML}{2A1B81}
\hypersetup{colorlinks,linkcolor=,urlcolor=links}

\newcommand{\fs}[2]{#2}

%\title[]{Sinusoidal Steady-State Analysis}
%\author[\textcolor{blue}{}]{\textcolor{violet}{System and Circuits. Topic-3: Electric Circuits}\\\vspace*{3mm}\textsf{{\small \textcolor{blue}{UC3M}}}
%}
%\date[07/04/2013]{{\small 07/04/2013}}
%\institute{\textcolor{white}{olmos@tsc.uc3m.es}}




\title[]{Sinusoidal Steady-State Analysis}
\author[\textcolor{blue}{Systems and Circuits}]{\textcolor{darkblue}{Pablo M. Olmos} (olmos@tsc.uc3m.es)\\ \textcolor{darkblue}{Emilio Parrado} (emipar@tsc.uc3m.es)}
\institute{\textcolor{white}{UC3M}}

\definecolor{darkblue}{rgb}{0.0, 0.0, 0.40}
\setbeamercolor{title}{fg=darkblue}
\setbeamercolor{frametitle}{fg=darkblue}
\definecolor{darkgreen}{rgb}{0.0, 0.4, 0.0}




%\logo{\includegraphics[scale=0.2]{Figures/uc3m-logo.pdf}}

\AtBeginSection[]
{
  \begin{frame}<beamer>{Index}
    \tableofcontents[currentsection,currentsubsection]
  \end{frame}
}

\begin{document}
\frame{
\titlepage
\thispagestyle{empty}
\begin{center}
\includegraphics[scale=0.05]{Figures/uc3m-logo.pdf}
\end{center}
}

\section{Introduction}

\frame{
\frametitle{Sinusoidal Steady-State Analysis, why?}
\begin{itemize}
\item Thus far, we have focused on circuits with constant sources. 
\item We are interested now in sources for which the value of the voltage or current varies sinusoidally.
\item Important area of study for several reasons:
\end{itemize}

\begin{block}{}
\begin{enumerate}
\item The generation, transmission, distribution and consumption of electric energy occur under essentially sinusoidal steady-sate conditions.
\item An understanding of sinusoidal behavior makes it possible to predict the behavior of circuits with non-sinusoidal sources.
\item Steady-state sinusoidal behavior often simplifies the design of electrical systems. 
\end{enumerate}
\end{block}


}


\frame{
\frametitle{The sinusoidal source}


\begin{columns}
\begin{column}{0.3\textwidth}
\begin{center}
\begin{circuitikz} \draw[black]
(0,0) to[sV=$v(t)$] (0,2);
\end{circuitikz}
\end{center}\end{column}
\begin{column}{0.5\textwidth}
\begin{align}\nn
v(t)=V_m\cos(\omega t+\phi).
\end{align}
\end{column}
\end{columns}

\begin{columns}
\begin{column}{0.3\textwidth}
\begin{center}
\begin{circuitikz} \draw[black]
(0,0) to[sI=$i(t)$] (0,2);
\end{circuitikz}
\end{center}
\end{column}
\begin{column}{0.5\textwidth}
\begin{align}\nn
i(t)=I_m\cos(\omega t+\phi).
\end{align}
\end{column}
\end{columns}

\begin{exampleblock}{}
\begin{itemize}
\item $V_m$ ($I_m)$ the amplitude.
\item $\omega$ is the angular frequency (rads/s). 
\item $T=\frac{2\pi}{\omega}$ is the period (seconds).
\item $\phi$  is the phase angle (rads).
\end{itemize}
\end{exampleblock}


}

\section{The sinusoidal response}

\frame{
\frametitle{The sinusoidal response: inductor}
\begin{align}\nn
v(t)=V_m\cos(\omega t+\phi).
\end{align}

\begin{center}
\begin{circuitikz} \draw[black]
(0,0) to[sV=$v(t)$] (0,2)
to[cspst](2,2)
to[R=$R$,i>=$i(t)$] (4,2)
to[L=$L$](4,0)
to[short] (0,0)
(1,2.5) node[]{$t=0$};
\end{circuitikz}
\end{center}

It can be proved that, for $t\rightarrow\infty$ (Sinusoidal Steady-State)
\begin{align}\nn
\lim_{t\rightarrow\infty} i(t)=\frac{V_m}{\sqrt{R^2+\omega^2L}}\cos(\omega t+\phi-\theta)\qquad \theta=\arctan(\frac{\omega L}{R}).
\end{align}


}

\frame{

\begin{exampleblock}{}
\begin{align}\nn
i(t)=\frac{V_m}{\sqrt{R^2+\omega^2L}}\cos(\omega t+\phi-\theta)\qquad \theta=\arctan(\frac{\omega L}{R}).
\end{align}
\end{exampleblock}

\begin{itemize}
\item $i(t)$ is also a sinusoidal function.
\item $i(t)$ has the same frequency that $v(t)$.
\item $i(t)$ and $v(t)$ only differ in the amplitude and the phase angle.
\end{itemize}


}


\frame{
\frametitle{The sinusoidal response: capacitor}
\begin{align}\nn
v(t)=V_m\cos(\omega t+\phi).
\end{align}

\begin{center}
\begin{circuitikz} \draw[black]
(0,0) to[sV=$v(t)$] (0,2)
to[cspst](2,2)
to[R=$R$,i>=$i(t)$] (4,2)
to[C=$C$](4,0)
to[short] (0,0)
(1,2.5) node[]{$t=0$};
\end{circuitikz}
\end{center}

It can be proved that, for $t\rightarrow\infty$ (Sinusoidal Steady-State)
\begin{align}\nn
\lim_{t\rightarrow\infty} i(t)=\frac{V_m}{\sqrt{R^2+\frac{1}{(\omega C)^2}}}\cos(\omega t+\phi+\theta)\qquad \theta=\arctan(\frac{1}{\omega RC}).
\end{align}


}

\frame{
\frametitle{The capacitor sinusoidal response (II)}

\begin{exampleblock}{}
\begin{align}\nn
i(t)=\frac{V_m}{\sqrt{R^2+\frac{1}{(\omega C)^2}}}\cos(\omega t+\phi+\theta)\qquad \theta=\arctan(\frac{1}{\omega RC}).
\end{align}
\end{exampleblock}

\begin{itemize}
\item $i(t)$ is also a sinusoidal function.
\item $i(t)$ has the same frequency that $v(t)$.
\item $i(t)$ and $v(t)$ only differ in the amplitude and the  phase angle.
\end{itemize}

}

\frame{
\frametitle{Sinusoidal Steady-State Analysis of passive circuits}

\begin{itemize}
\item Sinusoidal sources (all at the same frequency $\omega$).
\item Resistors, capacitors and inductors.
\item All magnitudes in the circuit vary according to a sinusoidal function!
\item \textbf{\textcolor{red}{We only have to compute the  amplitude and  phase angle of each voltage/current!}} 
\item We use the \textbf{phasor method}.
\end{itemize}





}

\section{The phasor method}

\frame{
\frametitle{The phasor}

\begin{block}{}
The phasor is a complex number that carries the amplitude and phase angle information about a sinusoidal function.
\end{block}

\begin{align}\nn
v(t)&=V_m\cos(\omega t+\phi)\Rightarrow \V=V_m\e^{j\phi}\\\nn\\\nn
i(t)&=I_a\cos(\omega t+\phi-\theta) \Rightarrow \I=I_a\e^{j(\phi-\theta)}\\\nn\\\nn
i(t)&=I_a\sin(\omega t+\phi)=I_a\cos(\omega t+\phi-\frac{\pi}{2}) \Rightarrow \I=I_a\e^{j(\phi-\frac{\pi}{2})}\\\nn\\\nn
%i(t)=\frac{V_m}{\sqrt{R^2+\frac{1}{(\omega C)^2}}}\cos(\omega t+\phi+\theta)& \Rightarrow \I=\frac{V_m}{\sqrt{R^2+\frac{1}{(\omega C)^2}}}\e^{j(\phi+\theta)}
\end{align}

}

\frame{
\frametitle{The resistor in the Sinusoidal Steady-State regime}

\begin{columns}
\begin{column}{0.3\textwidth}
\begin{center}
\begin{circuitikz} \draw[black]
(0,2) to[R=$R$,i>=\mbox{$i(t)=I_m\cos(\omega t+\phi)$},v_=$v(t)$] (0,0);
\end{circuitikz}
\end{center}\end{column}
\begin{column}{0.5\textwidth}
By Ohm's Law
\begin{align}\nn
v(t)=Ri(t)=R I_m \cos(\omega t+\phi).
\end{align}
\end{column}
\end{columns}

\vspace{1cm}
Therefore, in terms of their respective phasors:

\begin{exampleblock}{}
\begin{align}\nn
\V=R\I=R I_m \e^{j\phi}
\end{align}
\end{exampleblock}

}


\frame{
\frametitle{The Inductor in the Sinusoidal Steady-State regime}

\begin{columns}
\begin{column}{0.3\textwidth}
\begin{center}
\begin{circuitikz} \draw[black]
(0,2) to[L=$L$,i>=\mbox{$i(t)=I_m\cos(\omega t+\phi)$},v_=$v(t)$] (0,0);
\end{circuitikz}
\end{center}\end{column}
\begin{column}{0.5\textwidth}
\begin{align}\nn
v(t)=L\frac{di(t)}{dt}&=-\omega L I_m \sin(\omega t+\phi)\\\nn
&=-\omega L I_m \cos(\omega t+\phi-\frac{\pi}{2})
\end{align}
\end{column}
\end{columns}
\vspace{0.5cm}

Therefore, in terms of their respective phasors:

\begin{exampleblock}{}
\begin{align}\nn
\V=j\omega L \I
\end{align}
\end{exampleblock}

\begin{alertblock}{}
If we treat an inductor as a generalized resistor (called \textbf{impedance}) with value $\Z=j\omega L$, Ohm's law applies in the usual form for the phasors: $\V=\Z\I$.
\end{alertblock}

}

\frame{
\frametitle{The Capacitor in the Sinusoidal Steady-State regime}

\begin{columns}
\begin{column}{0.5\textwidth}
\begin{center}
\begin{circuitikz} \draw[black]
(0,0) to[C=$C$,i<=$i(t)$,v>=\mbox{$v(t)=V_m\cos(\omega t+\phi)$}] (0,2);
\end{circuitikz}
\end{center}\end{column}
\begin{column}{0.5\textwidth}
\begin{align}\nn
i(t)=C\frac{dv(t)}{dt}&=-\omega C V_m \sin(\omega t+\phi)\\\nn
&=-\omega C V_m \cos(\omega t+\phi-\frac{\pi}{2}).
\end{align}
\end{column}
\end{columns}
\vspace{0.5cm}

Therefore, in terms of their respective phasors:

\begin{exampleblock}{}
\begin{align}\nn
\I=j\omega C \V\Rightarrow \V=\frac{1}{j\omega C}\I.
\end{align}
\end{exampleblock}

\begin{alertblock}{}
If we treat the capacitor as a generalized resistor (called \textbf{impedance}) with value $\Z=\frac{1}{j\omega C}$, Ohm's law applies in the usual form for the phasors: $\V=\Z\I$.
\end{alertblock}

}

\frame{
\frametitle{Analysis of passive circuits in the Sinusoidal Steady-State regime}

\begin{itemize}
\item All the magnitudes of interest are sinusoidal of the same frequency $\omega$.
\item They are represented by phasors.
\item Resistors, inductors and capacitors are regarded as generalized resistors, called impedances, with respective values: $\Z_R=R$, $\Z_L=j\omega L$ and $\Z_C=\frac{1}{j\omega C}$. We measure impedance also in Ohms $\Omega$.
\item Ohm's law: $\V=\Z\I$. 
\item Although we don't prove it, \textbf{Kirchhoff's laws are also valid for phasors}.
\end{itemize}

\begin{block}{}
Passive circuits in the Sinusoidal Steady-State regime are analyzed using standard techniques: loop current method, voltage node method, Th\'evenin/Norton equivalent, ...
\end{block}

}

\section{Example one}

\frame{
Compute $i(t)$ using the phasor method if $v(t)=750\cos(5000 t+\frac{\pi}{6})$.
\begin{center}
\begin{circuitikz} \draw[black]
(0,0) to[sV=$v(t)$] (0,2)
to[R=$90\Omega$](2,2)
to[L=$32mH$] (4,2)
to[C=$5\mu F$,i>=$i(t)$](4,0)
to[short] (0,0);
\end{circuitikz}
\end{center}

\begin{exampleblock}{\textbf{Step 1}}
Compute the phasor associated to the source and the impedances:
\begin{itemize}
\item $v(t)=750\cos(5000 t+\frac{\pi}{6})\Rightarrow \V=750\e^{j\frac{\pi}{6}}$.
\item Resistor of 90 $\Omega\Rightarrow \Z_1=90 \Omega$.
\item Inductor of $32mH\Rightarrow \Z_2=j\omega L=j160\Omega$.
\item Capacitor of $5\mu F\Rightarrow \Z_3=\frac{1}{j\omega C}=-j40\Omega$.
\end{itemize}
\end{exampleblock}

}

\frame{

\begin{exampleblock}{\textbf{Step 2}}
We compute the phasor for the current $i(t)$ working with the phasor circuit.
\end{exampleblock}

\begin{center}
\begin{circuitikz}[european resistors] \draw[black]
(0,0) to[V=$\V$] (0,2)
to[R=$\Z_1$](2,2)
to[R=$\Z_2$] (4,2)
to[R=$\Z_3$,i>=$\I$](4,0)
to[short] (0,0);
\end{circuitikz}
\end{center}

Kirchhoff's voltage law:
\begin{align}\nn
&\I\Z_1+\I\Z_2+\I\Z_3-\V=90\I+j160\I-j40\I-\V=0\\\nn
&\Rightarrow \I=\frac{\V}{90+j120}=\frac{750\e^{j\frac{\pi}{6}}}{90+j120}
\end{align}
If we apply that $90+j120=150\e^{j0.927}$, then
\begin{align}\nn
\I=\frac{\V}{90+j120}=\frac{750\e^{j\frac{\pi}{6}}}{150\e^{j0.927}}=5\e^{-j0.4} A
\end{align}
}

\frame{
\begin{exampleblock}{\textbf{Step 3}}
Compute $i(t)$ from its phasor $\I$.
\end{exampleblock}

\begin{align}\nn
\I=5\e^{-j0.4}\Rightarrow i(t)=5\cos(5000 t-0.4)
\end{align}

\begin{alertblock}{Remark! Impedance in series}

\begin{center}
\begin{circuitikz} [european resistors] \draw[black]
(0,0) to[V=$\V$] (0,2)
to[R=$\Z_1$](2,2)
to[R=$\Z_2$] (4,2)
to[R=$\Z_3$,i>=$\I$](4,0)
to[short] (0,0)
(6,0) to[V=$\V$] (6,2)
to[short](7,2)
to[R,i>=$\I$] (7,0)
to[short] (6,0)
(9,1) node[]{$\Z_{eq}=\Z_1+\Z_2+\Z_3$};
\end{circuitikz}
\end{center}

As with resistors, the \textbf{equivalent impedance} of a set of impedances in series is equal to its sum. 

\end{alertblock}


}

\section{Example two}

\frame{
Compute $v(t),i_1(t), i_2(t)$ and $i_3(t)$ using the phasor method if $i_s(t)=8\cos(\omega t)$, where $\omega=2\times 10^5$ rads/s.
\begin{center}
\begin{circuitikz} \draw[black]
(-1,0) to[sI=$i(t)$] (-1,4)
(1,4) to[open,v<=$v(t)$] (1,0)
(2,0) to[R=$10\Omega$,i<=$i_1(t)$] (2,4)
(4,0) to[L=$40\mu H$,i<=$i_2(t)$] (4,2)
to [R=$6\Omega$] (4,4)
(6,0) to [C=$1\mu F$,i<=$i_3(t)$] (6,4)
to[short](-1,4)
(6,0) to[short] (-1,0);
\end{circuitikz}
\end{center}



}

\frame{

\begin{center}
\begin{circuitikz} \draw[black]
(-1,0) to[sI=$i(t)$] (-1,4)
(1,4) to[open,v<=$v(t)$] (1,0)
(2,0) to[R=$10\Omega$,i<=$i_1(t)$] (2,4)
(4,0) to[L=$40\mu H$,i<=$i_2(t)$] (4,2)
to [R=$6\Omega$] (4,4)
(6,0) to [C=$1\mu F$,i<=$i_3(t)$] (6,4)
to[short](-1,4)
(6,0) to[short] (-1,0);
\end{circuitikz}
\end{center}

\begin{exampleblock}{\textbf{Step 1}}
Compute the phasor associated to the source and the corresponding impedances:
\begin{itemize}
\item $i_s(t)=8\cos(\omega t)\Rightarrow \V=8\e^{j0}=8$.
\item Resistor of 10 $\Omega\Rightarrow \Z_1=10 \Omega$.
\item Resistor in series with Inductor of $4\mu H\Rightarrow \Z_2=6+j\omega L=6+j8\Omega$.
\item Capacitor of $1\mu F\Rightarrow \Z_3=\frac{1}{j\omega C}=-j5\Omega$.
\end{itemize}
\end{exampleblock}


}

\frame{

\begin{exampleblock}{\textbf{Step 2}}
We work with the phasor circuit.
\end{exampleblock}


\begin{center}
\begin{circuitikz}[european resistors] \draw[black]
(-1,0) to[I=$\I$] (-1,2)
(1,2) to[open,v<=$\V$] (1,0)
(2,0) to[R=$10\Omega$,i<=$\I_1(t)$] (2,2)
(4,0) to[R=$6+j8\Omega$,i<=$\I_2(t)$,-*] (4,2)
(6,0) to [R=$-5j\Omega$,i<=$\I_3(t)$] (6,2)
to[short](-1,2)
(6,0) to[short] (-1,0)
(4,2.5) node[] {$A$};
\end{circuitikz}
\end{center}

Kirchhoff's current law in node $A$:
\begin{align}\nn
\I=\I_1+\I_2+\I_3=\frac{\V}{\Z_1}+\frac{\V}{\Z_2}+\frac{\V}{\Z_3}=\frac{\V}{\Z_{eq}}.
\end{align}
The equivalent impedance of the set of three impedances in parallel is 
\begin{align}
\Z_{eq}=\left(\frac{1}{\Z_1}+\frac{1}{\Z_2}+\frac{1}{\Z_3}\right)^{-1},
\end{align}
i.e., the same as with simple resistors!!
}

\frame{

\begin{align}\nn
\Z_{eq}=&\left(\frac{1}{10}+\frac{1}{6+8j}+\frac{1}{-5j}\right)^{-1}=\left(0.1+\frac{6-8j}{100}+0.2j\right)^{-1}\\\nn&=\left(0.2\e^{j0.64}\right)^{-1}
=5\e^{-j0.64}.
\end{align}
Therefore
\begin{align}\nn
\V&=\Z_{eq}\I=40\e^{-j0.64}\rightarrow v(t)=40\cos(\omega t -0.64)\\\nn\\\nn
\I_1&=\frac{\V}{\Z_1}=4\e^{-j0.64}\rightarrow i_1(t)=4\cos(\omega t -0.64)\\\nn\\\nn
\I_2&=\frac{\V}{\Z_2}=\frac{40\e^{-j0.64}}{6+j8}=\frac{40\e^{-j0.64}}{10\e^{j0.927}}\approx 4\e^{-j0.5\pi}\rightarrow i_2(t)=4\cos(\omega t -0.5\pi)\\\nn\\\nn
\I_3&=\frac{\V}{\Z_3}=\frac{40\e^{-j0.64}}{-5j}=\frac{40\e^{-j0.64}}{5\e^{-j0.5\pi}}=8\e^{j0.932}\rightarrow i_3(t)=8\cos(\omega t+0.93).
\end{align}

}



\end{document}
