\documentclass[10pt]{beamer}

\mode<presentation>
{
 \usetheme{Boadilla}
\pagestyle{empty}

\setbeamerfont*{frametitle}{size=\normalsize,series=\bfseries}
\setbeamerfont*{block}{size=\normalsize,series=\bfseries}
%\setbeamertemplate{blocks}[rounded][shadow=true]
}


\usepackage{etex}
% \usepackage{helvet}
\usepackage{amsmath, amssymb}
\usepackage{color}
\usepackage{asymptote}
\usepackage{mathrsfs}
\usepackage{dsfont}
\usepackage{makeidx}
\usepackage{multido}
\usepackage{adjustbox}

\usepackage{pst-sigsys,pst-plot,pstricks-add}
\usepackage{pst-pdf}

\usepackage[american voltages, american currents,siunitx]{circuitikz}


\definecolor{links}{HTML}{2A1B81}
\hypersetup{colorlinks,linkcolor=,urlcolor=links}

\def\nn{\nonumber}

\definecolor{links}{HTML}{2A1B81}
\hypersetup{colorlinks,linkcolor=,urlcolor=links}

\newcommand{\fs}[2]{#2}

\title[]{Introduction to Electric Circuits. \\ Techniques of Circuit Analysis (II)}
\author[\textcolor{blue}{Systems and Circuits}]{\textcolor{darkblue}{Pablo M. Olmos} (olmos@tsc.uc3m.es)\\ \textcolor{darkblue}{Emilio Parrado} (emipar@tsc.uc3m.es)}
\institute{\textcolor{white}{UC3M}}

\definecolor{darkblue}{rgb}{0.0, 0.0, 0.40}
\setbeamercolor{title}{fg=darkblue}
\setbeamercolor{frametitle}{fg=darkblue}
\definecolor{darkgreen}{rgb}{0.0, 0.4, 0.0}


\AtBeginSection[]
{
  \begin{frame}<beamer>{Index}
    \tableofcontents[currentsection,currentsubsection]
  \end{frame}
}

\begin{document}
\frame{
\titlepage
\thispagestyle{empty}
\begin{center}
\includegraphics[scale=0.05]{Figures/uc3m-logo.pdf}
\end{center}
}

\section{Introduction}



\frame{

\begin{exampleblock}{}
Even though the node-voltage method and the loop-current methods are powerful techniques for solving circuits, we are still interested in methods that can be used to simplify circuits.
\end{exampleblock}

\begin{itemize}
\item Today: we provide new tools to simplify/transform a given circuit. 
\item Recall: we already know how to reduce a given circuit by merging resistors that are in parallel or in series.
\end{itemize}

}

\section{Source Transformations}

\frame{
\frametitle{Equivalent voltage/current sources}

\begin{block}{}
The two following configurations are equivalent \textbf{for any circuit we connect at the right to nodes $A$ and $B$}. Namely, none of the voltages/currents in the red circuit at the right of A-B will change.
\end{block}

\textbf{Configuration 1}
\begin{center}
\begin{circuitikz} \draw[black]
(-2,0) to[V=$V_{g}$] (-2,3)
(-2,3) to[R=$R_g$,-o](1,3)
(1,0) to[short,o-](-2,0);
\draw[red]
(1,3) to[short,i>=$I_A$](2,3)
(2,3) to[R=$R_2$,color=red](2,0)
(2,3) to[R=$R_3$,color=red](5,3)
(5,0) to[R=$R_4$,color=red](5,3)
to[short](6,3)
to[I=$I_s$,color=red](6,0)
(1,0) to[short](6,0)
(1,3.25) node[anchor=east] {$A$}
(1,-0.3) node[anchor=east] {$B$};
\end{circuitikz}
\end{center}

}

\frame{
\frametitle{Equivalent voltage/current sources}
\textbf{Configuration 2}
\begin{center}
\begin{circuitikz} \draw[black]
(-2,0) to[I=$I_g$] (-2,3)
(0,0) to[R=$R'_g$,i<=$I'_R$](0,3)
(0,3) to[short,-o](1,3)
(0,3) to[short] (-2,3)
(-2,0) to[short,-o] (1,0);
\draw[red]
(1,3) to[short,i>=$I_A$](2,3)
(2,3) to[R=$R_2$,color=red](2,0)
(2,3) to[R=$R_3$,color=red](5,3)
(5,0) to[R=$R_4$,color=red](5,3)
to[short](6,3)
to[I=$I_s$,color=red](6,0)
(1,0) to[short](6,0)
(1,3.25) node[anchor=east] {$A$}
(1,-0.3) node[anchor=east] {$B$};
\end{circuitikz}
\end{center}

where $I_g=\frac{V_g}{R_g}$ and $R'_g=R_g$.
}

\frame{
\frametitle{Equivalent voltage/current sources}

\textbf{Proof:}

\begin{itemize}
\item The red circuit does not see any difference between both configurations as long as  $V_A$-$V_B$ and $I_A$ are the same for both scenarios.
\item KVL in the left-most loop in configuration 1:
\begin{align}\nonumber
0=-V_g+I_A R_g+V_A-V_B \Rightarrow \color{blue}{V_g=I_A R_g+V_A-V_B;}
\end{align}
\item KCL in node $A$ in configuration 2:
\begin{align*}\nonumber
&I_g=I_R'+I_A\Rightarrow I_g=\frac{V_A-V_B}{R_g'}+I_A\\\nonumber
&\color{blue}{\Rightarrow R_g'I_g=(V_A-V_B)+I_A R_g'}
\end{align*}
\item If we compare the two expressions in blue: $V_g=I_gR_g'$ and $R_g=R_g'$.
\end{itemize}



}

\frame{
\begin{alertblock}{}
We have not assumed anything about the red circuit. This is true for any possible circuit that we connect to nodes $A$ and $B$.
\end{alertblock}

\begin{columns}
\begin{column}{0.5\textwidth}
\begin{center}
\begin{circuitikz} \draw[black]
(-2,0) to[V=$V_{g}$] (-2,3)
(-2,3) to[R=$R_g$,-o](1,3)
(1,0) to[short,o-](-2,0);
\draw[red]
(1,3) to[short,i>=$I_A$](2,3)
(2,3) to[generic,color=red](2,0)
to[short,color=red](1,0)
(1,3.25) node[anchor=east] {$A$}
(1,-0.3) node[anchor=east] {$B$};
\end{circuitikz}
\end{center}
\end{column}
\begin{column}{0.5\textwidth}
\begin{center}
\begin{circuitikz} \draw[black]
(-2,0) to[I=$\frac{V_g}{R_g}$] (-2,3)
(0,0) to[R=$R_g$](0,3)
(0,3) to[short,-o](1,3)
(0,3) to[short] (-2,3)
(-2,0) to[short,-o] (1,0);
\draw[red]
(1,3) to[short,i>=$I_A$](2,3)
(2,3) to[generic,color=red](2,0)
to[short,color=red](1,0)
(1,3.25) node[anchor=east] {$A$}
(1,-0.3) node[anchor=east] {$B$};
\end{circuitikz}
\end{center}
\end{column}
\end{columns}


}

\section{Th\'evenin and Norton Equivalent}

\frame{
\frametitle{Th\'evenin Equivalent}

\begin{exampleblock}{}
\begin{itemize}
\item Assume a circuit composed by independent voltage/current sources, dependent voltage/current sources and resistors.
\item For any two nodes $A$ and $B$ of the circuit, the circuit can be replaced by a voltage source  $V_{th}$ and a single  resistor $R_{th}$ in series configuration. This is called the \textbf{ Th�venin equivalent}.
\end{itemize}
\end{exampleblock}

%\begin{adjustbox}{scale=<0.5>} 

\begin{center}
\begin{circuitikz}[scale = 0.8, transform shape] \draw[black]
(-2,0) to[V=$V_{g}$] (-2,3)
(-2,3) to[R=$R_g$](0,3)
(1,0) to[short](-2,0)
(0,3) to[short](1,3)
(1,3) to[R=$R_2$](1,0)
(1,3) to[R=$R_3$](4,3)
(4,0) to[R=$R_4$](4,3)
to[short](5,3)
to[I=$I_s$,color](5,0)
(5,3) to[short,-o,i>=$I_A$](6,3)
(5,0) to[short,-o] (6,0)
(0,0) to[short](5,0)
(6.5,3.25) node[anchor=east] {$A$}
(6.5,-0.3) node[anchor=east] {$B$}

(8,0) to[V=$V_{th}$] (8,3)
(8,3) to[R=$R_{th}$,-o,i>=$I_A$](11,3)
(8,0) to[short,-o] (11,0)
(11.3,3.25) node[anchor=east] {$A$}
(11.3,-0.3) node[anchor=east] {$B$};
\end{circuitikz}
\end{center}

\begin{alertblock}{}
Any circuit that we connect to $A$ and $B$ does not see any difference. $V_A-V_B$ and $I_A$ are exactly the same in both cases.
\end{alertblock}

%\end{adjustbox}




}

\frame{
\frametitle{Th\'evenin equivalent. Computing $V_{th}$ and $R_{th}$}


\begin{exampleblock}{$V_{th}$}
$V_{th}$ is the voltage drop between $A$ and $B$ in \underline{open circuit}. Namely, we connect a resistor of $R_{open}=\infty\;\Omega$ between $A$ and $B$. Thus, $I_A=0$.
\end{exampleblock}


\begin{center}
\begin{circuitikz}[scale = 0.8, transform shape] \draw[black]
(-2,0) to[V=$V_{g}$] (-2,3)
(-2,3) to[R=$R_g$](1,3)
(1,0) to[short](-2,0);
\draw[black]
(1,3) to[short](2,3)
(2,3) to[R=$R_2$](2,0)
(2,3) to[R=$R_3$](5,3)
(5,0) to[R=$R_4$](5,3)
to[short](6,3)
to[I=$I_s$,color](6,0)
(6,3) to[short,-o,i>=$0$](7,3)
(6,0) to[short,-o] (7,0)
(1,0) to[short](6,0)
(7.5,3.25) node[anchor=east] {$A$}
(7.5,-0.3) node[anchor=east] {$B$}
(8,3) to[open, v=$V_{th}$] (8,0);
\end{circuitikz}
\end{center}


}


\frame{
\frametitle{Th\'evenin equivalent. Computing $V_{th}$ and $R_{th}$}


\begin{exampleblock}{$R_{th}$}
We place a short circuit across nodes $A$ and $B$ and calculate the resulting the short-circuit current $I_{sc}$. The resistor $R_{th}$ is computed as follows
\begin{align*}
R_{th}=\frac{V_{th}}{I_{sc}}
\end{align*}
\end{exampleblock}

\begin{center}
\begin{circuitikz}[scale = 0.8, transform shape] \draw[black]
(-2,0) to[V=$V_{g}$] (-2,3)
(-2,3) to[R=$R_g$](1,3)
(1,0) to[short](-2,0);
\draw[black]
(1,3) to[short](2,3)
(2,3) to[R=$R_2$](2,0)
(2,3) to[R=$R_3$](5,3)
(5,0) to[R=$R_4$](5,3)
to[short](6,3)
to[I](6,0)
(6,3) to[short,-o](7,3)
(6,0) to[short,-o] (7,0)
(1,0) to[short](6,0)
(7.5,3.25) node[anchor=east] {$A$}
(7.5,-0.3) node[anchor=east] {$B$}
(7,3) to[short,i>=$I_{sc}$] (7,0);
\end{circuitikz}
\end{center}

\begin{alertblock}{}
In the circuit above, what is the current across the resistor $R_4$?
\end{alertblock}
}


\frame{
\frametitle{Norton equivalent}

If we compute the Th\'evenin equivalent circuit of a given circuit, e.g.,

\begin{exampleblock}{}
\begin{center}
\begin{circuitikz}[scale = 0.6, transform shape] \draw[black]
(-2,0) to[V=$V_{g}$] (-2,3)
(-2,3) to[R=$R_g$](1,3)
(1,0) to[short](-2,0);
\draw[black]
(1,3) to[short](2,3)
(2,3) to[R=$R_2$](2,0)
(2,3) to[R=$R_3$](5,3)
(5,0) to[R=$R_4$](5,3)
to[short](6,3)
to[I=$I_s$,color](6,0)
(6,3) to[short,-o,i>=$I_A$](7,3)
(6,0) to[short,-o] (7,0)
(1,0) to[short](6,0)
(7.5,3.25) node[anchor=east] {$A$}
(7.5,-0.3) node[anchor=east] {$B$}

(9,0) to[V=$V_{th}$] (9,3)
(9,3) to[R=$R_{th}$,-o,i>=$I_A$](12,3)
(9,0) to[short,-o] (12,0)
(12.3,3.25) node[anchor=east] {$A$}
(12.3,-0.3) node[anchor=east] {$B$};
\end{circuitikz}
\end{center}
\end{exampleblock}

Then we can transform the the voltage source $V_{th}$ plus the resistor in series $R_{th}$ into a current source $I_N=V_{th}/R_{th}$ in parallel to a resistor $R_N=R_{th}$. This configuration is called the \textbf{Norton equivalent} of the circuit with respect nodes $A$ and $B$.



 \begin{exampleblock}{}
\begin{columns}
\begin{column}{0.5\textwidth}
\begin{center}
\begin{circuitikz}[scale = 0.6, transform shape] \draw[black]
(-2,0) to[I=$I_N$] (-2,3)
(0,0) to[R=$R_N$](0,3)
(0,3) to[short,-o](1,3)
(0,3) to[short] (-2,3)
(-2,0) to[short,-o] (1,0);
\end{circuitikz}
\end{center}
\end{column}
\begin{column}{0.5\textwidth}
\begin{align}\nonumber
I_N&=\frac{V_{th}}{R_{th}} \text{ (Short circuit current) }\\\nonumber
R_N&=R_{th}
\end{align}
\end{column}
\end{columns}
\end{exampleblock}

}


\section{Maximum power transfer theorem}


\frame{

\begin{itemize}
\item Circuit analysis plays an important role in the analysis of systems designed to transfer power from a source to a load.
\item In Communication and instrumentation systems are designed to transmit information via electro signals. The power available at the transmitter or the detector is typically limited.
\item Thus, transmitting as much of this power as possible to the receiver, or load, is critical.
\end{itemize}


}



\frame{

\frametitle{Maximum power transfer theorem}

\begin{columns}
\begin{column}{0.5\textwidth}
\begin{figure}
\centering\includegraphics[scale=1]{cellphone.jpg}
\end{figure}
\end{column}
\begin{column}{0.5\textwidth}
Mobile phone: Battery+Circuitry+Antenna. 
\end{column}
\end{columns}





Assume we compute the Th\'evenin equivalent of the first two and we model the antenna by a 
a resistor $R_a$.


\begin{center}
\begin{circuitikz}[scale = 0.8, transform shape]\draw
(9,0) to[V=$V_{th}$] (9,3)
(9,3) to[R=$R_{th}$,-o,i>=$I_A$](12,3)
(9,0) to[short,-o] (12,0)
(12.3,3.25) node[anchor=east] {$A$}
(12.3,-0.3) node[anchor=east] {$B$}
(12,3) to[R=$R_a$] (12,0)

(3,0) to[V=$V_{th}$] (3,3)
(3,3) to[R=$R_{th}$,-o](5,3)
(5.3,3.25) node[anchor=east] {$A$}
(5.3,-0.3) node[anchor=east] {$B$}
%(5,3) to[short] (5,1.7)
%(5,1.7) to[short] (6.35,1.7)
(6.75,1.5) node[anchor=east] {Antenna}
(5,3) to[generic] (5,0)
(5,0) to[short] (3,0);
\end{circuitikz}
\end{center}

}


\frame{

\begin{center}
\begin{circuitikz}[scale = 0.8, transform shape]\draw
(9,0) to[V=$V_{th}$] (9,3)
(9,3) to[R=$R_{th}$,-o,i>=$I_A$](12,3)
(9,0) to[short,-o] (12,0)
(12.3,3.25) node[anchor=east] {$A$}
(12.3,-0.3) node[anchor=east] {$B$}
(12,3) to[R=$R_a$] (12,0);
\end{circuitikz}
\end{center}

\begin{exampleblock}{}
The power provided by $V_{th}$ is $P=-V^2_{th}/(R_{th}+R_{a})$ Watt
\end{exampleblock}

\begin{alertblock}{}
The power dissipated in $R_{th}$ is 
\begin{align}\nonumber
P_{R_{th}}=\frac{V^2_{th}}{(R_{th}+R_{a})^2}R_{th} \text{ Watt }%\text{ Wasted energy (heat) }
\end{align}
\end{alertblock}

\begin{block}{}
The power dissipated in $R_{a}$ is 
\begin{align}\nonumber
P_{R_{a}}=\frac{V^2_{th}}{(R_{th}+R_{a})^2}R_{a}  \text{ Watt }
\end{align}
\end{block}

}

\frame{

Our goal is to design the antenna resistivity $R_a$ so that we maximize the power dissipated in $R_a$:

\begin{align}\nonumber
&\frac{\partial P_{R_{a}}}{\partial R_a}=\frac{V^2_{th}}{(R_{th}+R_{a})^2}-2R_a\frac{V^2_{th}}{(R_{th}+R_{a})^3}=0
\\
&\Rightarrow (R_{th}+R_{a})-2R_a=0\Rightarrow R_{th}=R_{a}\nonumber
\end{align}

\begin{block}{}
The power is maximized at the antenna if we design it to have a resistivity equal to $R_{th}$, the Th\'evenin resistor for the rest of the circuit.
\end{block}

}


\end{document}
